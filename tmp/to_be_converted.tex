\documentclass{article}

% Pacchetti utili
\usepackage[utf8]{inputenc} % Per la codifica dei caratteri
\usepackage[T1]{fontenc} % Per i font
\usepackage{graphicx} % Per le immagini
\usepackage{amsmath, amssymb} % Per la matematica
\usepackage{hyperref} % Per i link
\usepackage{natbib} % Per le citazioni
\usepackage{algorithm}
\usepackage{algorithmicx}
\usepackage{algpseudocode}
\usepackage{amsmath}
\usepackage{amsfonts}


\begin{document}

Gli \textit{item} e le \textit{query} vengono mappate su  vettore di \textit{embedding} in uno spazio comune $E = \mathbb{R}^k$. Normalmente lo spazio di \textit{embedding} ha una dimensione molto più piccola rispetto alla grandezza del corpus e cattura alcune strutture latenti dell'insieme di \textit{item} o \textit{query}. Gli elementi tra loro simili finiscono per essere vicini nello spazio di \textit{embedding}. Il concetto di "vicinanza" è definito da una misura di somiglianza.

\bibliographystyle{plain}
\bibliography{../bibliography}

\end{document}
