\documentclass{article}

% Pacchetti utili
\usepackage[utf8]{inputenc} % Per la codifica dei caratteri
\usepackage[T1]{fontenc} % Per i font
\usepackage{graphicx} % Per le immagini
\usepackage{amsmath, amssymb} % Per la matematica
\usepackage{hyperref} % Per i link
\usepackage{natbib} % Per le citazioni
\usepackage{algorithm}
\usepackage{algorithmicx}
\usepackage{algpseudocode}
\usepackage{amsmath}
\usepackage{amsfonts}
\usepackage{array}


\begin{document}

Il filtraggio basato sul contenuto utilizza le caratteristiche degli articoli per raccomandare altri articoli simili a quelli che l'utente piace, in base alle sue azioni precedenti o al feedback esplicito.

Per dimostrare il filtraggio basato sul contenuto, creiamo manualmente alcune caratteristiche per il Google Play Store. La figura seguente mostra una matrice delle caratteristiche, in cui ogni riga rappresenta un'app e ogni colonna rappresenta una caratteristica. Le caratteristiche potrebbero includere categorie (come Educazione, Casual, Salute), l'editore dell'app e molte altre. Per semplificare, supponiamo che questa matrice di caratteristiche sia binaria: un valore diverso da zero significa che l'app ha quella caratteristica.

Rappresenti anche l'utente nello stesso spazio delle caratteristiche. Alcune delle caratteristiche relative all'utente potrebbero essere fornite esplicitamente dall'utente. Ad esempio, un utente seleziona "App di intrattenimento" nel proprio profilo. Altre caratteristiche possono essere implicite, basate sulle app che l'utente ha precedentemente installato. Ad esempio, l'utente ha installato un'altra app pubblicata da Science R Us.

Il modello dovrebbe raccomandare articoli pertinenti a questo utente. Per farlo, è necessario prima scegliere una metrica di similarità (ad esempio, il prodotto scalare). Successivamente, bisogna impostare il sistema per assegnare un punteggio a ciascun articolo candidato in base a questa metrica di similarità. Si noti che le raccomandazioni sono specifiche per questo utente, poiché il modello non ha utilizzato alcuna informazione su altri utenti.

%###########################################



\bibliographystyle{plain}
\bibliography{../bibliography}

\end{document}
