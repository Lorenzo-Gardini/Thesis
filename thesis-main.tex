\documentclass[12pt,a4paper,openright,twoside]{book}
\usepackage{disi-thesis}

\school{\unibo}
\programme{Corso di Laurea Magistrale in Ingegneria e Scienze Informatiche}
\title{Progettazione e realizzazione di una libreria estendibile per la raccomandazione}
\author{Lorenzo Gardini}
\date{\today}
\subject{Data Mining}
\supervisor{Matteo Golfarelli}
\session{2}
\academicyear{2025-2026}

\mainlinespacing{1.241}

\begin{document}

\lstset{language=python}

\frontmatter\frontispiece

\begin{abstract}	
    La tesi esplora i principali approcci ai sistemi di raccomandazione, con focus su feedback espliciti e impliciti. Descrive poi l'attività di tirocinio presso l'azienda \textit{Data Reply}, dove è stata sviluppata una libreria Python per semplificare l'uso di modelli come \textit{Surprise}, \textit{LightFM} e \textit{Implicit}. L'obiettivo è rendere i sistemi di raccomandazione più accessibili e integrabili in ambito aziendale.
\end{abstract}

\begin{dedication}
A mia nonna.
\end{dedication}

\tableofcontents   
\listoffigures     
\lstlistoflistings

\mainmatter

\section{Introduzione}\label{introduzione}

Buco del culo di tu ma

\chapter{La recommendation}

\section{Introduzione alla recommendation}

Come fa \textit{Spotify} a sapere quale canzone l'utente vorrebbe ascoltare? E come fa \textit{Netflix} a suggerire la prossima serie da guardare? Questo è possibile grazie ad un sistema di \textit{recommendation} che, utilizzando tecniche di \textit{machine learning}, analizza cosa piace all'utente e gli propone contenuti su misura. 

Solitamente si utilizzano due tipi principali di raccomandazioni:

\begin{itemize}
    \item per la \textit{home page}: sono raccomandazioni personalizzate per un utente in base ai suoi interessi. Ogni utente vede articoli diversi
    \item su articoli correlati: sono raccomandazioni simili a un articolo specifico. Per esempio in \textit{Google Play}, gli utenti che visualizzano la pagina di un'app di matematica potrebbero vedere anche un riquadro con app correlate, come altre app di matematica o di scienze
\end{itemize}

Questo sistema aiuta gli utenti a scoprire contenuti interessanti all'interno di un'enorme quantità di dati. Per esempio su \textit{YouTube} ci sono miliardi di video, con nuovi contenuti che vengono aggiunti ogni giorno. Come può un utente trovare qualcosa di nuovo che valga la pena guardare o provare? La ricerca manuale è un'opzione. Tuttavia, un motore di \textit{recommendation} è in grado di suggerire contenuti che magari l'utente non avrebbe mai pensato di cercare da solo. Basti sapere che, secondo quanto dice la pagina di \href{https://developers.google.com/machine-learning/recommendation/overview}{Google Developers - Recommendation Systems Overview}:

\begin{itemize}
    \item il 40\% delle app installate dal \textit{Google Play} derivano da raccomandazioni
    \item il 60\% del tempo di visualizzazione su \textit{YouTube} proviene dalle raccomandazioni
\end{itemize}

Ci sono alcuni termini da introdurre:

\begin{itemize}
    \item \textit{sistema}: è il motore di \textit{recommendation} che, dato in input una \textit{query}, restituisce una lista ordinata di \textit{item} che si stima siano rilevanti o interessanti per l'utente in quel contesto
    \item \textit{item}: sono gli elementi/entità raccomandate dal sistema. Su \textit{Spotify} sono le canzoni, su \textit{Amazon} sono i prodotti, su \textit{Instagram} sono i \textit{post}.
    \item \textit{feedback}: sono le interazioni che legano l'utente agli \textit{item}, le informazioni pregresse con cui si costruisce il sistema. Possono essere espliciti (e.g. valutazioni, \textit{like}, etc.) o impliciti (e.g. numero visualizzazioni, acquistato o meno, etc.)
    \item \textit{query}: è una richiesta di raccomandazione, ovvero una domanda posta al sistema per ottenere suggerimenti personalizzati. Le \textit{query} possono essere una combinazione di informazioni dell'utente (e.g. ID, elementi con i quali ha interagito in passato, etc.) e contesto aggiuntivo (e.g. ora del giorno, per quanto tempo ha osservato quel prodotto o quell'episodio, etc.)
\end{itemize}

Per la creazione del sistema i passaggi tipici sono:

\begin{enumerate}
    \item collezione dei dati: raccolta dei feedback tra utenti e \textit{item} (e.g. valutazioni, click, acquisti, etc.)
    \item preprocessing dei dati: pulizia e trasformazione dei dati raccolti per renderli utilizzabili dal sistema di \textit{recommendation}
    \item creazione dei modello: addestramento di una serie di modelli di \textit{machine learning} sui dati pre-elaborati per apprendere le relazioni tra utenti e \textit{item}
\end{enumerate}

Una volta che i modelli sono addestrati, il sistema di \textit{recommendation} può essere utilizzato per generare raccomandazioni in tempo reale. Il processo tipico per generare raccomandazioni è il seguente:

\begin{enumerate}
    \item gli utenti interagiscono con il sistema a cui vengono sottomesse delle \textit{query}
    \item generazione dei candidati: per ogni \textit{query} si estrae un insieme di candidati da un corpus di \textit{item} potenzialmente enorme
    \item calcolo dello \textit{score}: a ciascun candidato viene assegnato un punteggio numerico
    \item \textit{re-ranking}: i candidati vengono ordinati per il punteggio ricevuto considerando eventuali ulteriori vincoli (e.g. rimuovendo contenuti che l'utente ha segnalato come non graditi, oppure aumentare il punteggio di contenuti più recenti). Il riordinamento può garantire maggiore varietà, attualità e imparzialità
\end{enumerate}

Per esempio un utente, accedendo alla sua \textit{home page} di \textit{Spotify}, genera una richiesta al sistema di \textit{recommendation} di \textit{Spotify}. Il sistema seleziona un sottoinsieme di canzoni che valuta interessanti per l'utente (per esempio potrebbe scegliere canzoni di artisti che l'utente sta ascoltando nell'ultimo periodo) e a ciascuna di esse assegna uno \textit{score}. Poi riordina la lista delle canzoni in modo decrescente rispetto allo score e seleziona le prime dieci canzoni da mostrare all'utente.

\section{Creazione del sistema}

Il sistema è composto da vari algoritmi che, interagendo in sequenza o in modo concorrente, svolgono i compiti sopra elencati. Per fare ciò si utilizzano algoritmi che, utilizzando i dati storici delle interazioni tra utenti e \textit{item} sono in grado di generare uno \textit{score} che indica quanto un \textit{item} è rilevante per un utente in un determinato contesto. Questi score possono essere delle previsioni, detti \textit{rating}, su quanto un utente apprezzerà un \textit{item} (per esempio il numero di "stelle" che l'utente metterebbe), su quanto è probabile che l'utente interagisca con quell'\textit{item} (per esempio 56\%) oppure un indicatore numerico puro che indica quanto è forte la relazione. Una volta ottenute queste previsioni, il sistema può ordinare gli \textit{item} in base agli score e presentare i risultati all'utente.

\section{Tipologie di feedback}

I sistemi di \textit{recommendation} si basano sull'analisi delle preferenze degli utenti, espresse attraverso due principali modalità: feedback esplicito e feedback implicito~\cite{ALS}. Ogni modalità presenta vantaggi, limitazioni e diverse tipologie di approcci. Poiché ciascuna forma di feedback fornisce informazioni diverse, un approccio ibrido è spesso preferibile. Ad esempio, nei sistemi che dispongono di valutazioni esplicite, il feedback implicito può essere utilizzato per arricchire il contesto dell'utente, migliorando le prestazioni del modello nei casi in cui i dati espliciti siano scarsi. I modelli basati sul \textit{Deep Learning} possono essere progettati per combinare diverse fonti, sfruttando il feedback implicito arricchito da attributi contestuali (tempo, luogo, dispositivo, ecc.) per ottimizzare la previsione.

\subsection{Feedback Esplicito}

Il feedback esplicito si riferisce a tutte quelle situazioni in cui l'utente comunica consapevolmente il proprio grado di interesse per un oggetto. Esempi includono le valutazioni da $1$ a $5$ stelle per i prodotti su \textit{Amazon} o il pollice su/giù per i video su \textit{YouTube} o semplicemente il \textit{like} su \textit{Instagram}. I modelli che utilizzano feedback espliciti predicono il \textit{rating} che un dato utente assegnerebbe ad un \textit{item}. Per esempio predicono che l'utente $u$ darebbe una votazione di $4$ su $5$ per la serie $i$.

Questi dati forniscono un segnale diretto, ma presentano però anche delle limitazioni:

\begin{itemize}
    \item più difficili da ottenere: richiedono che l'utente lasci intenzionalmente il suo riscontro
    \item rari nel mondo reale: molti utenti non lasciano mai valutazioni, dando origine a dati molto sparsi
    \item possono introdurre bias: ad esempio, utenti soddisfatti sono più propensi a lasciare valutazioni rispetto a quelli neutrali o insoddisfatti o viceversa
\end{itemize}

Nonostante ciò, il feedback esplicito rimane una delle fonti più affidabili per l'addestramento di modelli di \textit{recommendation}, in quanto consente di formulare il problema come una regressione delle valutazioni mancanti~\cite{Implicit_feedback}.

\subsubsection{Feedback Implicito}

Il feedback implicito, al contrario, non è fornito direttamente dall'utente, ma viene dedotto osservando i suoi comportamenti. Esempi tipici includono:

\begin{itemize}
    \item cronologia degli acquisti
    \item cronologia di navigazione
    \item pattern di ricerca
    \item tempo di permanenza su una pagina
    \item interazioni come click, visualizzazioni o movimenti del mouse
    \item acquisto o meno di un prodotto
\end{itemize}

I feedback impliciti possono essere raccolti facilmente in modo automatico e passivo senza che l'utente debba interagire intenzionalmente. Inoltre, sono molto più abbondanti rispetto alla controparte esplicita. Forniscono una rappresentazione più completa del comportamento degli utenti, specialmente in contesti in cui il feedback esplicito è assente o insufficiente. Allo stato attuale, i modelli di \textit{Deep Learning} moderni utilizzano grandi quantità di feedback impliciti. Questi modelli non forniscono un \textit{rating} ma uno \textit{score} numerico per la coppia utente-\textit{item}. Questo numero può essere la probabilità che l'utente interagisca con quell'\textit{item} oppure un numero puro che da solo non ha alcun significato, ma che messo in relazione con altri \textit{score} calcolati per lo stesso utente su \textit{item} diversi, permette di ordinare gli \textit{item}.

L'utilizzo di dati impliciti, però, può creare complicazioni:

\begin{itemize}
    \item ambiguità: un'azione (es. visualizzazione di un contenuto) non implica necessariamente una preferenza positiva
    \item rumore nei dati: molte interazioni potrebbero essere accidentali o non intenzionali
    \item assenza di segnali negativi chiari: è difficile distinguere tra mancanza di interesse e mancata esposizione all'oggetto
\end{itemize}

\section{Approcci alla recommendation}

Tutti gli algoritmi di raccomandazione, sia quelli impiegati nella \textit{generazione dei candidati} sia quelli utilizzati nella fase di \textit{ranking}, si basano generalmente su una delle seguenti categorie di dati:

\begin{itemize}
    \item \textit{Content-based filtering}, che utilizza le caratteristiche esplicite degli utenti e/o degli item
    \item \textit{Collaborative filtering}, che si basa esclusivamente sulle interazioni tra utenti e item
    \item \textit{Approcci ibridi}, che combinano informazioni sia di tipo \textit{content-based} che \textit{collaborative} per migliorare le prestazioni del sistema di raccomandazione
\end{itemize}

In generale per ogni algoritmo si possono definire le seguenti caratteristiche:

\begin{itemize}
    \item feedback: il tipo di feedback utilizzato (esplicito o implicito)
    \item tecnica: la tecnica utilizzata per generare le raccomandazioni (e.g. \textit{Matrix Factorization}, \textit{Deep Learning}, etc.)
    \item approccio: \textit{model-based} o \textit{memory-based}
    \item filtering: il tipo di approccio utilizzato (\textit{content-based}, \textit{collaborative})
\end{itemize}

Ogni modello potrebbe avere ibridi nelle varie caratteristiche, per esempio un modello \textit{Deep Learning} che utilizza la tecnica della \textit{Matrix Factorization}.


\subsection{Content-based filtering}
Il \textit{content-based filtering} utilizza la similarità tra gli \textit{item} per raccomandare elementi simili a quelli che l'utente apprezza o ha apprezzato. Ad esempio, se l'utente $A$ guarda due film \textit{fantasy}, il sistema può raccomandargli altri film dello stesso genere. L'idea è che se ad un utente è piaciuto un certo \textit{item} con certe caratteristiche, il sistema proporrà altri \textit{item} simili per contenuto. Per prima cosa si crea un profilo degli \textit{item}, cioè viene rappresentato tramite un insieme di caratteristiche (\textit{feature}). Per un film, per esempio, si possono utilizzare il/i generi, gli attori principali, l'anno di uscita, la durata etc. Dopodiché si costruisce un profilo dell'utente analizzando gli \textit{item} che ha valutato positivamente. Il profilo rappresenta una media delle caratteristiche degli \textit{item} preferiti. Si confronta il profilo dell'utente con gli \textit{item} non ancora visti usando una metrica di similarità (es. coseno, distanza euclidea). Vengono raccomandati gli \textit{item} più simili al profilo dell'utente.

\subsubsection{Esempio}

Si supponga che all'utente $A$ siano piaciuti due film:

\begin{itemize}
    \item \textit{Matrix} i cui generi sono \textit{Azione} e \textit{Fantascienza}
    \item \textit{Inception} i cui generi sono \textit{Azione}, \textit{Fantascienza} e \textit{Thriller}
\end{itemize}

Si crea il profilo dell'utente con la media delle caratteristiche: 

\begin{itemize}
    \item \textit{Azione} $= 1$
    \item \textit{Fantascienza} $= 1$
    \item \textit{Thriller} $= 0.5$
\end{itemize}

Poi si selezionano altri film dal catalogo, ad esempio:

\begin{itemize}
    \item \textit{Interstellar} i cui generi sono \textit{Fantascienza} e \textit{Drammatico}
    \item \textit{John Wick} il cui genere è \textit{Azione}
\end{itemize}

\begin{figure}[htbp]
    \centering
    \includegraphics[scale=0.5]{figures/content_based_filtering.PNG}
    \caption{La figura mostra i vettori delle caratteristiche dei film (in questo caso il genere) per i film in esempio. Se il film appartiene ad uno specifico genere viene posto il valore $1$, altrimenti $0$}
    \label{fig:item_vector}
\end{figure}

Si creano vettori per i film considerando tutti i generi analizzati fino ad ora e si calcola la similarità con il profilo dell'utente. Si ottiene, utilizzando per esempio la similarità coseno, un valore di $0.47$ per $Interstellar$ e di $0.67$ per \textit{John Wick}, quindi quest'ultimo viene raccomandato prima di \textit{Interstellar}, perché è più simile al profilo dell'utente.

\subsubsection{Vantaggi e svantaggi}

I vantaggi di quest'approccio sono:
\begin{itemize}
    \item il modello non ha bisogno di dati su altri utenti, poiché le raccomandazioni sono specifiche per questo utente. Questo lo rende più facile da scalare a un grande numero di utenti
    \item il modello può catturare gli interessi specifici di un utente e può raccomandare \textit{item} di nicchia che interessano a pochissimi altri utenti
\end{itemize}

Gli svantaggi di quest'approccio sono:
\begin{itemize}
    \item poiché la rappresentazione delle caratteristiche degli oggetti è in parte progettata manualmente, questa tecnica richiede molta conoscenza del dominio. Il modello può essere solo quindi valido quanto le caratteristiche progettate manualmente
    \item il modello può fare raccomandazioni solo basate sugli interessi esistenti dell'utente. In altre parole, ha una capacità limitata di espandere gli interessi dell'utente
\end{itemize}

\subsection{Collaborative filtering}
Il \textit{collaborative filtering} (\textit{CF}) viene introdotto per la prima volta dal sistema \textit{Tapestry}~\cite{Tapestry} e si riferiva al fatto che \textit{"people collaborate to help one another perform the filtering process in order to handle the large amounts of email and messages posted to newsgroups"}. Questo termine ha poi assunto significati più ampi. In senso generale, indica il processo di filtraggio di informazioni o pattern tramite tecniche che coinvolgono la collaborazione tra più utenti, agenti e fonti di dati. Il \textit{CF} esiste in molte forme e, sin dalla sua introduzione, sono stati proposti numerosi metodi basati su di esso. Per affrontare alcune delle limitazioni del \textit{content-base filtering}, il \textit{collaborative filtering} utilizza simultaneamente le somiglianze tra utenti e \textit{item} per fornire raccomandazioni. I modelli possono raccomandare un \textit{item} all'utente $A$ in base agli interessi di un utente simile $B$ anche se $A$ non ha mai visto \textit{item} simili. In generale, le tecniche di \textit{CF} possono essere suddivise in tre categorie: \textit{CF} basato sulla memoria, \textit{CF} basato su modelli e i modelli ibridi~\cite{Su}. Tra le tecniche rappresentative basate sulla memoria ci sono i metodi \textit{nearest neighbor-based}, come il \textit{CF} basato sugli utenti (\textit{user-based}) e quello basato sugli oggetti (\textit{item-based})~\cite{Sarwar}. I modelli a fattori latenti, come la \textit{Matrix Factorization}, sono esempi di \textit{CF} basato su modelli. Il \textit{CF} basato sulla memoria presenta delle limitazioni nella gestione di dati sparsi e su larga scala, poiché calcola la similarità basandosi sugli oggetti in comune. I metodi basati su modelli sono diventati più popolari grazie alla loro maggiore efficacia nel gestire la sparsità e la scalabilità. Molti approcci \textit{CF} basati su modelli possono essere estesi con reti neurali, permettendo modelli più flessibili e scalabili grazie all'accelerazione computazionale offerta dal \textit{Deep Learning}~\cite{Zhang}. In generale, il \textit{CF} utilizza i dati di interazione tra utenti e oggetti per fare previsioni e raccomandazioni.


\subsubsection{Esempio}

Si supponga di avere una matrice di interazioni tra utenti e \textit{item}, in cui ogni riga corrisponde ad un utente e ogni colonna corrisponde ad un \textit{item}. Ogni elemento della matrice rappresenta il feedback dell'utente sull'\textit{item}, ad esempio una valutazione numerica o un'indicazione binaria di interesse.

L'idea principale del modello è la seguente:

\begin{itemize}
    \item calcolare la similarità tra gli utenti (o alternativamente tra gli \textit{item}) utilizzando la similarità coseno
    \item utilizzare queste similarità per prevedere il grado di interesse di un utente verso un \textit{item} non ancora valutato, basandosi sui voti dati da utenti simili
\end{itemize}

La similarità coseno tra due vettori \( \mathbf{x}, \mathbf{y} \in \mathbb{R}^n \) è definita come:

\[
\text{sim}(\mathbf{x}, \mathbf{y}) = \cos(\theta) = \frac{\mathbf{x} \cdot \mathbf{y}}{\|\mathbf{x}\|_2 \|\mathbf{y}\|_2}
\]

Nel contesto del modello, ad esempio, i vettori \( \mathbf{x} \) e \( \mathbf{y} \) rappresentano le rispettive righe della matrice di interazione, ossia le valutazioni degli utenti sui diversi film. Per prevedere la valutazione che l'utente potrebbe dare ad un film si considera la media delle valutazioni date dagli utenti al film pesata dalla similarità tra l'utente iniziale e gli altri.

Si supponga di voler raccomandare un film all'utente \textit{Verde}, in particolare prevedere il punteggio che potrebbe assegnare al film \textit{Fight Club}, che non ha ancora visto.

\begin{figure}[htbp]
    \centering
    \includegraphics[scale=0.5]{figures/collaborative_filtering/interaction_matrix.png}
    \caption{matrice di interazione tra utenti e film. Le righe rappresentano gli utenti, le colonne i film e gli elementi della matrice i voti che l'utente ha dato al film. Se l'utente non ha visto il film, l'elemento della matrice è vuoto.}
\end{figure}

Si calcola la similarità coseno tra l'utente \textit{Verde} e gli altri utenti che hanno visto il film. 

Similarità tra Verde e Blu:

\[
\text{sim}(Verde, Blu) = \frac{4 \cdot 3 + 0 \cdot 4.5 + 5 \cdot 5 + 4 \cdot 0 + 0 \cdot 1.5}{\sqrt{4^2 + 0 + 5^2 + 4^2 + 0} \cdot \sqrt{3^2 + 4.5^2 + 5^2 + 0 + 1.5^2}} \approx 0.65
\]

Similarità tra Verde e Giallo:

\[
\text{sim}(Verde, Giallo) = \frac{4 \cdot 0 + 0 \cdot 0 + 5 \cdot 0 + 4 \cdot 3 + 0 \cdot 4}{\sqrt{57} \cdot \sqrt{0 + 0 + 0 + 9 + 16}} \approx 0.319
\]

Ora, la predizione per l'utente \textit{Verde} per \textit{Fight Club} si calcola come media pesata:

\[
\frac{0.65 \times 1.5 + 0.32 \times 4}{0.65 + 0.32} \approx 2.33
\]

Quindi si prevede che l'utente verde valuterebbe il film \textit{Fight Club} con un voto intorno a 2.33.

Il modello basato sulla similarità coseno è semplice da implementare e intuitivo, ma può soffrire di problemi come la scarsità di dati e l'effetto del rumore nelle valutazioni. Tuttavia, rimane un punto di partenza valido per sistemi di raccomandazione collaborativi. Se si considerassero solamente i contribuiti dei $k$ utenti più simili, si potrebbe migliorare la robustezza della previsione. Questo approccio è noto come \textit{K-Nearest Neighbors} (\textit{KNN}).

\subsubsection{Vantaggi e svantaggi}

I vantaggi di quest'approccio sono:

\begin{itemize}
  \item non è necessaria conoscenza del dominio: nel \textit{CF}, la raccomandazione si basa esclusivamente sulle valutazioni o interazioni passate degli utenti con gli \textit{item}, senza utilizzare informazioni aggiuntive sulle caratteristiche degli stessi. Questo significa che il modello non richiede conoscenza specifica del dominio (ad esempio il genere di un film o le preferenze dettagliate degli utenti), ma si limita a individuare pattern di comportamento simili tra utenti o \textit{item}. Di conseguenza, il sistema è applicabile in diversi contesti in modo generale e automatico, apprendendo le affinità direttamente dai dati di interazione.
  \item serendipità: anche se il modello non sa che l'utente è interessato a un determinato \textit{item} potrebbe comunque raccomandarlo perché utenti simili sono interessati a quell'\textit{item}

\end{itemize}

Gli svantaggi di quest'approccio sono:

\begin{itemize}
  \item non si può gestire \textit{item} nuovi: il \textit{CF} classico fatica a gestire \textit{item} nuovi, cioè quelli che non hanno ancora ricevuto valutazioni o interazioni dagli utenti (problema del \textit{cold start}). Senza dati storici il modello non può calcolare similarità né fare previsioni attendibili
  \item difficoltà nell'includere caratteristiche aggiuntive per \textit{query}/\textit{item}: le caratteristiche aggiuntive (\textit{side features}) sono tutte quelle informazioni oltre all'id dello user o dell'\text{item}. Per esempio, per i film, le caratteristiche possono includere il paese o l'età. Includere queste caratteristiche migliora la qualità del modello
\end{itemize}

\subsubsection{Approcci}

Esistono due approcci principali per facilitare tale confronto, che costituiscono le due tecniche fondamentali del \textit{collaborative filtering}: 

\begin{itemize}
    \item approcci \textit{neighborhood}: si concentrano sulle relazioni tra \textit{item} oppure, alternativamente, tra utenti. Un approccio \textit{item}-\textit{item} modella la preferenza di un utente per un \textit{item} sulla base delle valutazioni di \textit{item} simili da parte dello stesso utente
    \item modelli basati sulla \textit{Matrix Factorization}: un approccio in cui sia gli \textit{item} che gli utenti vengono proiettati in uno stesso spazio latente, cercando di spiegare le valutazioni osservate attraverso fattori latenti inferiti automaticamente dai feedback
\end{itemize}

\section{Matrix Factorization}\label{matrix_factorization}

La \textit{Matrix Factorization} è una tecnica utilizzata per rappresentare una matrice come prodotto di due o più matrici. Consente di estrarre automaticamente strutture latenti dai dati, rendendo possibile la scoperta di relazioni implicite tra entità \cite{MC}. Questa tecnica è alla base di molte applicazioni in ambiti diversi, tra cui l'elaborazione di segnali, la compressione dei dati, la visione artificiale e, in particolare, i sistemi di \textit{recommendation}.

Formalmente, data una matrice $R \in \mathbb{R}^{m \times n}$, la fattorizzazione mira a trovare due matrici $W \in \mathbb{R}^{m \times k}$ e $H \in \mathbb{R}^{n \times k}$ tali che:
\[
R \approx WH^T
\]
dove 
\begin{itemize}
    \item le righe di $W$ corrispondono agli \textit{embedding} degli utenti
    \item le righe di $H$ corrispondono agli \textit{embedding} degli \textit{item}
    \item $k \ll \min(m,n)$ è il rango latente scelto
\end{itemize}

\begin{figure}[htbp]
    \centering
    \includegraphics[scale=0.5]{figures/collaborative_filtering/matrix_factorization.PNG}
    \caption{rappresentazione della matrice $R$ come prodotto delle due matrici $W$ e $H$}
    \label{fig:matrix_factorization}
\end{figure}

Questa approssimazione riduce la dimensionalità dei dati, semplifica il modello e cattura le relazioni principali presenti nella matrice originaria. Un celebre esempio dell'efficacia di questa tecnica nei sistemi di \textit{recommendation} è il \textit{Netflix Prize} del 2006. Il team vincente la utilizzò per migliorare le previsioni di rating del 10\% rispetto al sistema originario di Netflix \cite{TheNP}. I principali vantaggi nel suo utilizzo sono la scalabilità, poiché i modelli sono efficienti da memorizzare e computare, e la capacità di generalizzazione, in quanto riescono a catturare relazioni latenti non esplicitamente osservate. Tuttavia, esistono anche alcuni limiti, tra cui il problema della \textit{cold start}, che rende difficile raccomandare per nuovi utenti o nuovi \textit{item}, e la sparsità, che può portare a una una bassa qualità delle raccomandazioni\cite{SVD_analysis}. Pur con alcune limitazioni, essa costituisce la base per molti degli algoritmi di \textit{recommendation} più efficaci oggi in uso, ed è spesso integrata con approcci più complessi, come i modelli \textit{Deep Learning} o i grafi.

\subsection{Concetto di Spazio Latente ed Embedding}
Nel contesto della \textit{Matrix Factorization}, uno dei concetti centrali è lo spazio latente. Questo spazio rappresenta un insieme di dimensioni astratte, non osservabili direttamente, ma che spiegano le correlazioni nei dati. Ad esempio, in un sistema di raccomandazione di film, utenti e film possono essere rappresentati come vettori in uno spazio $k$-dimensionale, dove ogni dimensione può rappresentare una caratteristica latente come la preferenza per il genere "azione", la complessità della trama, il livello di drammaticità, ecc. Queste dimensioni non sono etichettate esplicitamente, ma vengono apprese automaticamente durante l'addestramento. Questa rappresentazione vettoriale in uno spazio latente prende il nome di \textit{embedding}. Ogni utente e ogni \textit{item} (ad esempio, film, prodotto o articolo) viene associato a un vettore numerico che cattura le sue proprietà implicite in modo compatto.

\subsection{Apprendimento delle Feature}
Le feature apprese nel contesto della matrix factorization non sono predefinite, ma emergono come risultato dell'ottimizzazione. In pratica, si cerca di minimizzare una funzione di perdita, tipicamente la somma dei quadrati degli errori tra le valutazioni osservate e quelle predette:

\[
\min_{U,V} \sum_{(i,j) \in \mathcal{K}} (R_{ij} - W_i \cdot H_j^T)^2 + \lambda ( \|W_i\|^2 + \|H_j\|^2 )
\]

dove:
\begin{itemize}
    \item $\mathcal{K}$ è l'insieme delle coppie $(i,j)$ per cui $R_{ij}$ è noto;
    \item $\lambda$ è un termine di regolarizzazione che evita l'overfitting.
\end{itemize}

Durante l'ottimizzazione, i vettori $W_i$ e $H_j$ vengono aggiornati in modo tale da minimizzare l'errore sulle osservazioni note, catturando implicitamente le caratteristiche principali che influenzano le preferenze. I modelli apprendono automaticamente un vettore di \textit{embedding} per ciascun utente che spieghi al meglio le sue preferenze. Di conseguenza, gli \textit{embedding} di utenti con gusti simili risulteranno vicini tra loro. Allo stesso modo il modello apprende gli \textit{embedding} dei 
\textit{item} in modo da spiegare al meglio la matrice di feedback

\subsection{Esempio}

Per esempio si consideri un sistema di \textit{recommendation} di film in cui i dati di addestramento consistono in una matrice di feedback in cui:

\begin{itemize}
    \item Ogni riga rappresenta un utente
    \item Ogni colonna rappresenta un \textit{item} (un film)
    \item Il feedback sui film rientra in una delle due categorie:
    \begin{itemize}
        \item esplicito: gli utenti specificano quanto gli è piaciuto un determinato film fornendo una valutazione numerica
        \item implicito: se un utente guarda un film, il sistema deduce che l'utente è interessato
    \end{itemize}
\end{itemize}

Per semplificare, si supponga che la matrice di feedback sia binaria; cioè il simbolo "check" verde indica interesse per il film.

Quando un utente visita la homepage, il sistema dovrebbe raccomandare film basati su:

\begin{enumerate}
    \item somiglianza con i film che l'utente ha apprezzato in passato
    \item film che utenti simili hanno apprezzato
\end{enumerate}

Il modello, applicando la tecnica di \textit{Matrix Factorization} ottiene in maniera automatica una rappresentazione degli utenti e dei film in uno spazio di \textit{embedding} a bassa dimensione. In questo spazio, gli utenti e i film sono rappresentati da vettori che catturano le loro caratteristiche latenti. Per esempio il modello potrebbe apprendere due dimensioni latenti per i film: la prima esprime se il film è per adulti o per bambini mentre la seconda esprime il grado in cui il film è un \textit{blockbuster} (film molto popolare e di grande successo al botteghino) o un film d'autore. Stessa cosa succede per gli utenti: il modello apprende le preferenze degli utenti in relazione a queste due dimensioni latenti. Si può pensare allo spazio di \textit{embedding} come a una rappresentazione astratta comune sia agli \textit{item} che agli utenti.

\begin{figure}[htbp]
    \centering
    \includegraphics[scale=0.4]{figures/collaborative_filtering/2D_matrix.PNG}
    \caption{matrice che mostra i film guardati dagli utenti, le categorie dei film e le preferenze degli utenti estese}
    \label{fig:2D_matrix}
\end{figure}


La previsione del gradimento di un utente per un film si ottiene calcolando il prodotto scalare tra l'embedding dell'utente e quello del film.

\[
r_{ui} = w_u \cdot h_i
\]

Se si volesse calcolare il gradimento dell'utente \textit{Giallo} per il film \textit{Kung Fu Panda}, si calcolerebbe il prodotto scalare tra i rispettivi \textit{embedding}:

\[
\begin{bmatrix}
0.1 & 1
\end{bmatrix}
\times
\begin{bmatrix}
1 \\
-1
\end{bmatrix} = -0.9
\]

quindi il modello prevede che l'utente \textit{Giallo} non apprezzi il film \textit{Kung Fu Panda}.

\subsection{Vantaggi e svantaggi}
I vantaggi della \textit{Matrix Factorization} includono:
\begin{itemize}
    \item scalabilità: la \textit{Matrix Factorization} può gestire matrici molto grandi e sparse, tipiche dei sistemi reali
    \item generalizzazione: le feature apprese permettono di effettuare predizioni su dati mai visti prima (\textit{cold-start}), anche se con alcune limitazioni
    \item compattezza: gli \textit{embedding} permettono una rappresentazione compatta ma espressiva delle entità
\end{itemize}

Gli svantaggi della \textit{Matrix Factorization} includono:

\begin{itemize}
    \item ottimizzazione complessa: l'addestramento può essere computazionalmente costoso e sensibile all'inizializzazione e ai parametri di regolarizzazione
    \item scarsa interpretabilità: gli \textit{embedding} latenti non sono facilmente interpretabili, rendendo difficile spiegare le raccomandazioni agli utenti
\end{itemize}
\chapter{Algoritmi}\label{algoritmi}

\section{Notazione}\label{notazione}

La notazione utilizzata è la seguente:

\begin{itemize}
    \item \textit{user}: l'utente
    \item \textit{item}: l'oggetto
    \item \textit{rating}: valutazione/i
    \item $m$: numero di \textit{user}
    \item $n$: numero di \textit{item}
    \item $R$: l'insieme di tutti i \textit{rating}.
    \item $R_{train}$, $R_{test}$ e $\hat{R}$ indicano il set di addestramento, il set di test e l'insieme dei \textit{rating} previsti.
    \item $U$ : l'insieme di tutti gli \textit{user}. $u$ e $v$ indicano gli \textit{user}.
    \item $I$ : l'insieme di tutti gli \textit{item}. $i$ e $j$ indicano gli \textit{item}.
    \item $U_i$ : l'insieme di tutti gli \textit{user} che hanno valutato l'\textit{item} $i$.
    \item $U_{ij}$ : l'insieme di tutti gli \textit{user} che hanno valutato sia l'\textit{item} $i$ che l'\textit{item} $j$.
    \item $I_u$ : l'insieme di tutti gli \textit{item} valutati dallo \textit{user} $u$.
    \item $I_{uv}$ : l'insieme di tutti gli \textit{item} valutati sia dallo \textit{user} $u$ che dallo \textit{user} $v$.
    \item $r_{ui}$ : il \textit{rating} \textit{vero} dello \textit{user} $u$ per l'\textit{item} $i$.
    \item $\hat{r}_{ui}$ : il \textit{rating} \textit{stimato} dello \textit{user} $u$ per l'\textit{item} $i$.
    \item $b_{ui}$ : il \textit{rating} di base dello \textit{user} $u$ per l'\textit{item} $i$.
    \item $\mu$ : la media di tutti i \textit{rating}.
    \item $\mu_u$ : la media di tutti i \textit{rating} dati dallo \textit{user} $u$.
    \item $\mu_i$ : la media di tutti i \textit{rating} date all'\textit{item} $i$.
    \item $\sigma_u$ : la deviazione standard di tutti i \textit{rating} dati dallo \textit{user} $u$.
    \item $\sigma_i$ : la deviazione standard di tutte le valutazioni date all'\textit{item} $i$.
    \item $N_i^k(u)$ : i $k$ vicini più prossimi dello \textit{user} $u$ che hanno valutato l'\textit{item} $i$. Questo insieme è calcolato utilizzando una metrica di similarità.
    \item $N_u^k(i)$ : i $k$ vicini più prossimi dell'\textit{item} $i$ che sono valutati dallo \textit{user} $u$. Questo insieme è calcolato utilizzando una metrica di similarità.
\end{itemize}

\section{Algoritmi per il feedback esplicito}\label{algoritmi-per-feedback-esplicito}

\subsection{Matrix Factorization}

\subsubsection{SVD}\label{svd}

L'algoritmo SVD (\textit{singular value decomposition}), è stato reso popolare da Simon Funk durante la competizione \textit{the Netflix Prize} dimostrando come modelli di fattorizzazione matriciale sono superiori alle tecniche classiche basate su \textit{nearest neighbor}\ref{knn} per
la produzione di raccomandazioni.

I modelli di \textit{matrix factorization} mappano gli \textit{user} e gli \textit{item} in uno spazio latente comune di dimensione $k$, che rappresenta il numero di caratteristiche latenti. Ogni \textit{item} $i$ è associato a un vettore $q_i$ di dimensione $k$, che misura quanto l'\textit{item} possieda ciascuna di queste caratteristiche latenti. Per ogni \textit{user} $u$, invece, il vettore $p_u$ misura l'interesse dello \textit{user} per gli \textit{item}. Il numero di fattori è un iper-parametro dell'algoritmo.

In questo spazio, le interazioni tra \textit{user} e \textit{item} vengono modellate come prodotti scalari tra i rispettivi vettori. Lo spazio latente cerca di spiegare i \textit{rating} caratterizzando sia gli \textit{item} che gli textit{user} in base a fattori che vengono automaticamente dedotti. Ad esempio, se gli \textit{item} sono film, i fattori potrebbero appresentare il genere piuttosto che un altro (e.g. Azione contro Drama), profondità della trama o il concetto di "adatto ai bambini".

Il prodotto scalare risultante cattura l'interazione tra lo \textit{user} $u$ e l'\textit{item} $i$, che corrisponde all'interesse complessivo dello \textit{user} per le caratteristiche dell'\textit{item}. Il \textit{rating} finale viene ottenuto aggiungendo anche i predittori di base sopra menzionati, che dipendono solo dallo \textit{user} o dall'\textit{item}. Pertanto, un \textit{rating} viene predetto dalla regola~\cite{SVD_analysis}~\cite{Recommendation_book}:

\[
\hat{r}_{ui} = \mu + b_u + b_i + q_i^T p_u
\]

Dove:
\begin{itemize}
    \item $ \hat{r}_{ui} $ è il \textit{rating} previsto per
    l'\textit{item} $i$ da parte dello \textit{user} $u$.
    \item $ \mu $ è la media generale delle \textit{rating}.
    \item $ b_u $ e $ b_i $ sono i bias dello \textit{user} $u$ e dell'\textit{item} $i$ rispettivamente. Sono una sorta di correzione basata sull'effetto dello \textit{user} e dell'\textit{item}.
    \item $ q_i^T p_u $ è il prodotto interno tra i vettori latenti dello \textit{user} e dell'\textit{item}.
\end{itemize}

Se lo \textit{user} $u$ è sconosciuto, allora il bias $b_u$ e i fattori $p_u$ vengono considerati uguali a zero. Lo stesso vale per
l'\textit{item} $i$, con $b_i$ e $q_i$ anch'essi assunti uguali a zero.

Per apprendere i parametri del modello ($b_u$, $b_i$, $p_u$, $q_i$), si minimizza l'errore quadratico regolarizzato tra le \textit{rating} reali e quelle previste. L'errore quadratico è dato da:

\[
\min \sum\limits_{(u,i) \in K} \left( (r_{ui} - \hat{r}_{ui})^2 + \lambda (\|q_i\|^2 + \|p_u\|^2 + b_u^2 + b_i^2) \right)
\]


Dove il primo termine è l'errore quadratico tra le \textit{rating} previste e reali e il secondo termine è la regolarizzazione, che penalizza valori troppo grandi per i parametri $b_u$, $b_i$, $p_u$, $q_i$ per evitare l'overfitting.

Per ottimizzare questi parametri, viene usata la \textit{stochastic gradient descent} (SGD), che aggiorna i parametri dopo ogni  esempio di training (ad esempio, per ogni \textit{rating} di un \textit{user}).

Per ogni \textit{rating} ($r_{ui}$) data, viene fatta una previsione ($\hat{r}_{ui}$), e l'errore di previsione associato ($e_{ui} = r_{ui} - \hat{r}_{ui}$) viene calcolato. Per un dato caso di addestramento ($r_{ui}$), modifichiamo i parametri spostandoci nella
direzione opposta al gradiente, ottenendo:

\begin{itemize}
    \item $b_u \leftarrow b_u + \gamma \cdot e_{ui}$
    \item $b_i \leftarrow b_i + \gamma \cdot e_{ui}$
    \item $q_i \leftarrow q_i + \gamma \cdot e_{ui} \cdot p_u$
    \item $p_u \leftarrow p_u + \gamma \cdot e_{ui} \cdot q_i$
\end{itemize}

dove $\gamma$ è il \textit{learning rate}. $\gamma$ e $\lambda$ sono iperparametri dell'algoritmo.

Queste formule vengono utilizzate per aggiornare i parametri durante l'addestramento del modello, in modo da ridurre l'errore tra le
\textit{rating} reali e quelle previste.

Per ottenere un ulteriore miglioramento, possono applicare $\gamma$ e $\lambda$ separati per i bias degli \textit{user}, i bias degli
\textit{item} e i fattori stessi~\cite{SVD_optimized}.

I punti di forza dell'algoritmo sono:

\begin{itemize}
    \item semplicità: l'algoritmo SVD è relativamente semplice da comprendere e implementare.
    \item riduzione della dimensionalità: l'algoritmo permette di ridurre la dimensione del problema mappando sia gli \textit{user} che gli \textit{item} in uno spazio latente di dimensione inferiore, gestendo la sparsità delle matrici. Funziona molto bene quando la matrice delle valutazioni è abbastanza completa. 
    \item caratteristiche latenti: identifica strutture sottostanti che non sono immediatamente evidenti.
\end{itemize}

L'algoritmo soffre anche di diverse problematiche:

\begin{itemize}
    \item problemi con la sparsità: può produrre raccomandazioni imprecise quando la matrice delle valutazioni è troppo sparsa, perché la decomposizione non riesce a estrarre informazioni significative. 
    \item non tiene conto di informazioni aggiuntive: non considera altri fattori come informazioni temporali, contenuti aggiuntivi sugli \textit{item} o preferenze esplicite/implicite dello \textit{user} che non sono registrati nella matrice.
    \item computazionalmente costosa: la decomposizione di una matrice grande è costosa in termini di tempo e risorse.
    \item overfitting: se non adeguatamente regolarizzato si rischia l'overfitting.
\end{itemize}

\subsubsection{SVD\protect++}

La precisione delle previsioni può essere migliorata considerando anche il feedback implicito, che fornisce un'indicazione aggiuntiva delle preferenze degli \textit{user}. Questo è particolarmente utile per gli \textit{user} che hanno fornito molto più feedback implicito che esplicito. Anche nei casi in cui il feedback implicito indipendente è assente, è possibile catturare un segnale significativo tenendo conto degli \textit{item} che gli \textit{user} hanno valutato, indipendentemente dal valore del \textit{rating}. Ciò ha portato a diversi metodi (\textit{Asymmetric-SVD}, \textit{SVD++}, \textit{SVD\_KNN}, ecc.~\cite{SVD++, SVD_KNN}) che modellano il fattore \textit{user} in base agli \textit{item} valutati. Il metodo \textit{SVD++} ha dimostrato di offrire una precisione superiore rispetto a \textit{SVD}.

Viene aggiunto un secondo set di fattori degli \textit{item}, che collega ogni \textit{item} $i$ a un vettore di fattori $y_i$ di dimensione $k$. Questi nuovi fattori vengono utilizzati per caratterizzare gli \textit{user} in base al set di \textit{item} che hanno valutato. La nuova predizione si calcola come segue:

\[
\hat{r}_{ui} = \mu + b_u + b_i + q_i^T \left(p_u + |I_u|^{\frac{1}{2}} \sum\limits_{j \in I_u} y_j \right)
\]

Ora, un \textit{user} $u$ viene modellato come $p_u + |I_u|^{\frac{1}{2}} \sum\limits_{j \in I_u} y_j$, mentre la parte $\sum\limits_{j \in I_u} y_j$ rappresenta i feedback impliciti. Poiché i vettori $y_j$ sono centrati intorno a zero grazie alla regolarizzazione $|I_u|^{\frac{1}{2}}$, la varianza rispetto all'intervallo di valori osservati $|I_u|$ viene stabilizzata.

I parametri del modello vengono determinati minimizzando la funzione di errore quadratico regolarizzato, utilizzando sempre \textit{stochastic gradient descent}. Si itera su tutti i \textit{rating}:

\begin{itemize}
  \item $b_u \leftarrow b_u + \gamma \cdot (e_{ui} - \lambda \cdot b_u)$
  \item $b_i \leftarrow b_i + \gamma \cdot (e_{ui} - \lambda \cdot b_i)$
  \item $q_i \leftarrow q_i + \gamma \cdot \left( e_{ui} \cdot \left( p_u + |I_u|^{-\frac{1}{2}} \sum\limits_{j \in I_u} y_j \right) - \lambda \cdot q_i \right)$
  \item $p_u \leftarrow p_u + \gamma \cdot (e_{ui} \cdot q_i - \lambda \cdot p_u)$
  \item $\forall j \in I_u: \quad y_j \leftarrow y_j + \gamma \cdot \left( e_{ui} \cdot |I_u|^{-\frac{1}{2}} \cdot q_i - \lambda \cdot y_j \right)$
\end{itemize}

È possibile introdurre diversi tipi di feedback implicito nel modello simultaneamente, utilizzando set aggiuntivi di fattori degli \textit{item}.

I punti di forza dell'algoritmo sono:

\begin{itemize}
    \item miglioramento della personalizzazione: SVD++ è un miglioramento significativo rispetto a SVD, poiché prende in considerazione anche l'influenza degli \textit{item} che lo \textit{user} ha già valutato nel termine $\sum\limits_{j \in I_u} y_j$, il che lo rende molto più sensibile alle preferenze individuali dello \textit{user}.
    \item migliore gestione della sparsità: poiché SVD++ tiene conto dei feedback impliciti, riesce a fare previsioni migliori anche quando la matrice delle valutazioni è sparsa.
    \item previsioni più accurate: i feedback impliciti aiutano a produrre previsioni più accurate, soprattutto in scenari dove gli \textit{user} hanno interagito con più \textit{item}.
\end{itemize}

L'algoritmo soffre anche di diverse problematiche:

\begin{itemize}
    \item complessità computazionale maggiore: la necessità di aggiornare $y_i$ aumenta il carico computazionale.
    \item richiede più dati: poiché prende in considerazione anche i feedback impliciti, ha bisogno di un numero maggiore di dati per generare previsioni precise.
    \item overfitting: SVD++ è più propenso a overfitting su dataset piccoli.
\end{itemize}

\subsubsection{NMF}

Un algoritmo di \textit{collaborative filtering} basato sulla \textit{fattorizzazione matriciale non negativa}.  

Questo algoritmo è molto simile a SVD\ref{svd} ma con una restrizione che tutti gli elementi devono essere non negativi. Questo ha senso, ad esempio, quando si tratta di rating o quantità (che non possono essere negativi).

L'idea è approssimare la matrice $R$ come prodotto di due matrici più piccole:

\[
R \approx WH
\]

dove:
\begin{itemize}
    \item $W \in \mathbb{R}_{\geq 0}^{m \times k}$ rappresenta la matrice dei profili latenti degli \textit{user};
    \item $H \in \mathbb{R}_{\geq 0}^{k \times n}$ rappresenta la matrice dei profili latenti degli \textit{item};
    \item $k$ è il numero di fattori latenti. Anche in questo caso è iper-parametro dell'algoritmo
\end{itemize}

Il \textit{rating} stimato dello \textit{user} $u$ per l'\textit{item} $i$ è calcolato come~\cite{NMF2}~\cite{NMF3}:

\[
\hat{r}_{ui} = \sum_{f=1}^k W_{uf} H_{fi} = q_i^T p_u
\]

dove i fattori utente e articolo vengono mantenuti positivi

L'obiettivo di apprendimento è minimizzare l'errore quadratico sui \textit{rating} osservati nel set di addestramento $R_{train}$:

\[
\min_{W, H} \sum_{(u,i) \in R_{train}} \left( r_{ui} - \hat{r}_{ui} \right)^2 \quad \text{s.t.} \quad W \geq 0,\ H \geq 0
\]

In forma matriciale, questo si può esprimere come:

\[
\min_{W, H} \ \| R_{train} - WH \|_F^2 \quad \text{con} \quad W \geq 0,\ H \geq 0
\]

dove $\| \cdot \|_F$ è la norma di Frobenius.

La procedura di ottimizzazione è una SGD regolarizzata~\cite{NMF} con una scelta specifica della dimensione del passo che garantisce la non negatività dei fattori, a condizione che anche i loro valori iniziali siano positivi.

A ogni iterazione i fattori vengono aggiornati come segue:

\begin{equation}
    \begin{split}
        p_{uf} &\leftarrow p_{uf} \cdot \frac{\sum\limits_{i \in I_u} q_{if} \cdot r_{ui}}{\sum\limits_{i \in I_u} q_{if} \cdot \hat{r}_{ui} + \lambda_u |I_u| p_{uf}}\\
        q_{if} &\leftarrow q_{if} \cdot \frac{\sum\limits_{u \in U_i} p_{uf} \cdot r_{ui}}{\sum\limits_{u \in U_i} p_{uf} \cdot \hat{r}_{ui} + \lambda_i |U_i| q_{if}}
    \end{split}
\end{equation}

Questo algoritmo è altamente dipendente dai valori iniziali con cui vengono inizializzate le matrici $H$ e $W$. I fattori latenti degli \textit{user} e degli \textit{item} vengono inizializzati casualmente in modo uniforme tra un minimo e un massimo, solitamente nell'intervallo $[0, 1]$.

Vengono introdotti due nuovi iper-parametri $\lambda_u$ e $\lambda_i$ che corrispondono alla regolarizzazione rispettivamente per \textit{user} e \textit{item}.

Anche in questo caso si può utilizzare la predizione con l'utilizzo di baseline\ref{svd}.

\[
\hat{r}_{ui} = \mu + b_u + b_i + q_i^T p_u
\]

garantendo comunque fattori positivi. Le baselines sono ottimizzate nello stesso modo dell'algoritmo SVD. Pur producendo una migliore accuratezza, la versione che utilizza baseline sembra molto incline all'overfitting, che può essere ridotto diminuendo $k$ o aumentando la regolarizzazione.

NFM, grazie al vincolo della non negatività, è più interpretabile di SVD: i valori nelle matrici fattorizzate $W$ e $H$ sono tutti $\geq 0$, quindi possono essere interpretati come pesi o intensità (es. quanto un utente apprezza un certo genere, quanto un item rappresenta un tema).

L'algoritmo soffre anche di diverse problematiche:

\begin{itemize}
    \item dipendenza dall’inizializzazione: l’algoritmo può convergere a minimi locali diversi a seconda dei valori iniziali, e non garantisce una soluzione unica o ottima
    
    \item non adatta per dati con valori negativi: NMF non può gestire valori negativi nei dati di input, a differenza di SVD
    
    \item convergenza più lenta: rispetto ad altri metodi la convergenza può essere più lenta e richiede tuning di più parametri
    \item mancanza di soluzione chiusa: SVD può gestire valori negativi e ha una soluzione ottima in termini di errore quadratico minimo
\end{itemize}

\subsection{KNN (K-Nearest Neighbors)}\label{knn}
Si tratta di algoritmi derivati direttamente da un approccio di base basato sui \textit{nearest neighbors}.

Gli algoritmi ispirati al KNN (\textit{K Nearest Neighbors}) sono una classe di algoritmi di raccomandazione che si basano sull'idea che gli \textit{user} simili tendono a valutare gli stessi \textit{item} in modo simile. Questi algoritmi sono semplici da implementare e possono essere molto efficaci per problemi di raccomandazione a piccola scala.

Il numero effettivo di vicini che vengono considerati per calcolare la predizione è minore o uguale a $k$: potrebbero non esserci abbastanza vicini e/o gli insiemi $N_i^k(u)$ e $N_u^k(i)$ includono solo vicini per i quali la misura di similarità è positiva (non avrebbe senso considerare \textit{user} o \textit{item} correlati negativamente).

$k$ è un iperparametro di ciascun algoritmo.

Alcune misure di similarità, sia per \textit{user} che per \textit{item}, sono:
\begin{itemize}
    \item Coseno:
        \[
        \text{cosine sim}(u, v) = \frac{\sum\limits_{i \in I_{uv}} r_{ui} \cdot r_{vi}}{\sqrt{\sum\limits_{i \in I_{uv}} r_{ui}^2} \cdot \sqrt{\sum\limits_{i \in I_{uv}} r_{vi}^2}}
        \]
        oppure
        \[
        \text{cosine sim}(i, j) = \frac{\sum\limits_{u \in U_{ij}} r_{ui} \cdot r_{uj}}{\sqrt{\sum\limits_{u \in U_{ij}} r_{ui}^2} \cdot \sqrt{\sum\limits_{u \in U_{ij}} r_{uj}^2}}
        \]
    \item \textit{Mean Square Difference} (MSD):
        \[
        \text{msd}(u, v) = \frac{1}{|I_{uv}|} \cdot \sum\limits_{i \in I_{uv}} (r_{ui} - r_{vi})^2
        \]
        oppure
        \[
        \text{msd}(i, j) = \frac{1}{|U_{ij}|} \cdot \sum\limits_{u \in U_{ij}} (r_{ui} - r_{uj})^2
        \]
        La similarità è calcolata come:
        \begin{align*}
            \text{msd sim}(u, v) &= \frac{1}{\text{msd}(u, v) + 1} \\
            \text{msd sim}(i, j) &= \frac{1}{\text{msd}(i, j) + 1}
        \end{align*}
        Il termine $+1$ viene aggiunto per evitare divisioni per zero.
    \item Pearson: Il coefficiente di correlazione di Pearson può essere visto come una similarità del coseno centrato sulla media. Se non ci sono \textit{item} comuni, la similarità è 0 (non -1).
        \[
        \text{pearson sim}(u, v) = \frac{\sum\limits_{i \in I_{uv}} (r_{ui} - \mu_u) \cdot (r_{vi} - \mu_v)}{\sqrt{\sum\limits_{i \in I_{uv}} (r_{ui} - \mu_u)^2} \cdot \sqrt{\sum\limits_{i \in I_{uv}} (r_{vi} - \mu_v)^2}}
        \]
        oppure
        \[
        \text{pearson sim}(i, j) = \frac{\sum\limits_{u \in U_{ij}} (r_{ui} - \mu_i) \cdot (r_{uj} - \mu_j)}{\sqrt{\sum\limits_{u \in U_{ij}} (r_{ui} - \mu_i)^2} \cdot \sqrt{\sum\limits_{u \in U_{ij}} (r_{uj} - \mu_j)^2}}
        \]
    \item Pearson con baseline \cite{Recommendation_book}: calcola il coefficiente di correlazione di Pearson (ridotto) tra tutte le coppie di \textit{user} (o \textit{item}) utilizzando le baseline anziché le medie. Il parametro di riduzione aiuta a evitare l'overfitting quando sono disponibili solo poche valutazioni. Se non ci sono \textit{item} comuni, la similarità è 0 (non -1). Introduce un nuovo iperparametro che corrisponde alla riduzione (o \textit{``shrinkage''}). Se impostato uguale a 0, non viene applicata nessuna riduzione.
        \[
        \text{pearson baseline sim}(u, v) = \hat{\rho}_{uv} = \frac{\sum\limits_{i \in I_{uv}} (r_{ui} - b_{ui}) \cdot (r_{vi} - b_{vi})}{\sqrt{\sum\limits_{i \in I_{uv}} (r_{ui} - b_{ui})^2} \cdot \sqrt{\sum\limits_{i \in I_{uv}} (r_{vi} - b_{vi})^2}}
        \]
        oppure
        \[
        \text{pearson baseline sim}(i, j) = \hat{\rho}_{ij} = \frac{\sum\limits_{u \in U_{ij}} (r_{ui} - b_{ui}) \cdot (r_{uj} - b_{uj})}{\sqrt{\sum\limits_{u \in U_{ij}} (r_{ui} - b_{ui})^2} \cdot \sqrt{\sum\limits_{u \in U_{ij}} (r_{uj} - b_{uj})^2}}
        \]
        Il coefficiente ridotto si calcola come:
        \begin{align*}
            \text{pearson baseline shrunk sim}(u, v) &= \frac{|I_{uv}| - 1}{|I_{uv}| - 1 + \text{shrinkage}} \cdot \hat{\rho}_{uv} \\
            \text{pearson baseline shrunk sim}(i, j) &= \frac{|U_{ij}| - 1}{|U_{ij}| - 1 + \text{shrinkage}} \cdot \hat{\rho}_{ij}
        \end{align*}
        Per il calcolo della baseline, si consideri la parte di \ref{knn_baseline}.
\end{itemize}

Altro iper-parametro da considerare è il supporto minimo, che corrisponde al numero minimo di \textit{item} in comune o il numero minimo di \textit{user} in comune affinché la similarità non sia zero, i.e., se $|I_{uv}| < \text{min support}$, allora $\text{sim}(u, v) = 0$. Lo stesso vale per gli \textit{item}.

\subsubsection{KNN base}

L'algoritmo KNN base è l'algoritmo più semplice. Prevede il  \textit{rating} di un \textit{user} $u$ per un \textit{item} $i$ come la media ponderata dei \textit{rating} degli $k$ vicini più simili di $u$ o $i$, a seconda che si utilizzi un approccio basato sugli \textit{user} o sugli \textit{item}.

La predizione viene calcolata come:

\[
\hat{r}_{ui} = \frac{\sum\limits_{v \in N^k_i(u)} \text{sim}(u, v) \cdot r_{vi}}{\sum\limits_{v \in N^k_i(u)} \text{sim}(u, v)}
\]

oppure

\[
\hat{r}_{ui} = \frac{\sum\limits_{j \in N^k_u(i)} \text{sim}(i, j) \cdot r_{uj}}{\sum\limits_{j \in N^k_u(i)} \text{sim}(i, j)}
\]

dipendentemente dall'approccio utilizzato.

\subsubsection{KNN con la media}
\label{algoritmo-knn-con-la-media}

L'algoritmo è una variante dell'algoritmo KNN base che tiene conto della media dei \textit{rating} degli \textit{user} o degli \textit{item}.

La predizione viene calcolata come:

\[
\hat{r}_{ui} = \mu_u + \frac{\sum\limits_{v \in N_k(u)} \text{sim}(u, v) \cdot (r_{vi} - \mu_v)}{\sum\limits_{v \in N_k(u)} \text{sim}(u, v)}
\]

oppure

\[
\hat{r}_{ui} = \mu_i + \frac{\sum\limits_{j \in N^k_u(i)} \text{sim}(i, j) \cdot (r_{uj} - \mu_j)}{\sum\limits_{j \in N^k_u(i)} \text{sim}(i, j)}
\]

dipendentemente dall'approccio utilizzato.

\subsubsection{KNN normalizzato}
\label{algoritmo-knn-normalizzato}

L'algoritmo è una variante dell'algoritmo che utilizza la media con l'aggiunta della normalizzazione \textit{z-score}, con la deviazione standard dello \textit{user} o dell'\textit{item}, dei \textit{rating} corrispondenti prima di calcolare la similarità.

La predizione viene calcolata come:

\[
\hat{r}_{ui} = \mu_u + \sigma_u \frac{\sum\limits_{v \in N^k_i(u)} \text{sim}(u, v) \cdot (r_{vi} - \mu_v) / \sigma_v}{\sum\limits_{v \in N^k_i(u)} \text{sim}(u, v)}
\]

oppure

\[
\hat{r}_{ui} = \mu_i + \sigma_i \frac{\sum\limits_{j \in N^k_u(i)} \text{sim}(i, j) \cdot (r_{uj} - \mu_j) / \sigma_j}{\sum\limits_{j \in N^k_u(i)} \text{sim}(i, j)}
\]

dipendentemente dall'approccio utilizzato.

\subsubsection{KNN con baseline}\label{knn_baseline}
\label{knn-con-baseline}

L'algoritmo KNN con baseline \cite{KNN_baseline} è una variante dell'algoritmo base che tiene conto degli effetti di bias degli \textit{user} o degli \textit{item}.

La predizione viene calcolata come:

\[
\hat{r}_{ui} = b_{ui} + \frac{\sum\limits_{v \in N^k_i(u)} \text{sim}(u, v) \cdot (r_{vi} - b_{vi})}{\sum\limits_{v \in N^k_i(u)} \text{sim}(u, v)}
\]

oppure

\[
\hat{r}_{ui} = b_{ui} + \frac{\sum\limits_{j \in N^k_u(i)} \text{sim}(i, j) \cdot (r_{uj} - b_{uj})}{\sum\limits_{j \in N^k_u(i)} \text{sim}(i, j)}
\]

dipendentemente dall'approccio utilizzato.

La baseline $b_{ui}$ viene calcolata come:

\[
b_{ui} = \mu + b_u + b_i
\]

Per calcolare $b_u$ e $b_i$, occorre minimizzare il seguente errore quadratico regolarizzato:

\[
\sum\limits_{r_{ui} \in R_{train}} \left(r_{ui} - \mu + b_u + b_i\right)^2 + \lambda \left(b_u^2 + b_i^2 \right).
\]

Il termine di regolarizzazione $\lambda \left(b_u^2 + b_i^2 \right)$ serve per evitare l'overfitting penalizzando la grandezza dei parametri.

La minimizzazione può essere effettuata tramite \textit{Stochastic Gradient Descent} o \textit{Alternating Least Squares}.

Per \textit{Stochastic Gradient Descent}, si introduce $\lambda$ per la regolarizzazione, $\gamma$ il \textit{learning rate} e $N$ il numero di epoche.

Per \textit{Alternating Least Squares}, i due valori di $b_u$ e $b_i$ si ottengono come:

\[
b_i = \frac{\sum\limits_{r_{ui} \in R_{train}} (r_{ui} - \mu)}{\lambda_2 + |U_i|}
\]

e

\[
b_u = \frac{\sum\limits_{r_{ui} \in R_{train}} (r_{ui} - \mu - b_i)}{\lambda_3 + |I_u|}
\]

e si introducono come iperparametri $\lambda_1$ e $\lambda_2$ per la regolarizzazione e $N$ per il numero di epoche.

I punti di forza della famiglia di algoritmi KNN sono:
\begin{itemize}
    \item Semplicità: Gli algoritmi KNN sono facili da capire e implementare. L'idea di base di trovare ``vicini'' simili è intuitiva e facilmente comprensibile.
    \item Nessuna assunzione sui dati: l'algoritmo non fa assunzioni sulla distribuzione dei dati. Questo lo rende flessibile e adatto a una varietà di dataset.
    \item Aggiornamento: l'aggiunta di nuovi dati non richiede una fase di addestramento esplicita. Questo lo rende utile in ambienti in cui i dati cambiano frequentemente o per previsioni online.
    \item Flessibilità: KNN può essere utilizzato sia per problemi di raccomandazione basati su \textit{user} che basati su \textit{item}.
\end{itemize}

Gli algoritmi soffrono anche di diverse problematiche:
\begin{itemize}
    \item Costo computazionale: KNN può essere computazionalmente costoso, soprattutto con set di dati di grandi dimensioni.
    \item Requisiti di memoria: KNN richiede la memorizzazione dell'intero set di dati, il che può essere problematico per set di dati molto grandi.
    \item Sensibilità alla scelta di k: La scelta del valore di k può avere un impatto significativo sulle prestazioni di KNN. Un valore di k troppo piccolo può portare a un overfitting, mentre un valore di k troppo grande può portare a un underfitting. Inoltre, la ricerca dei $k$ vicini più prossimi richiede il calcolo delle distanze tra tutti i punti dati.
    \item Gestione della sparsità dei dati: la sparsità può rendere difficile trovare vicini significativi e quindi portare a raccomandazioni di bassa qualità.
\end{itemize}

\subsection{CoClustering}\label{coclustering}

La soluzione proposta da Thomas George e Srujana Merugu~\cite{Co-Clustering} utilizza il \textit{co-clustering}. Questa tecnica viene utilizzata per raggruppare simultaneamente due entità in un dataset. Nel caso di un sistema di \textit{recommendation} basato su \textit{collaborative filtering}, l'obiettivo è trovare gruppi di \textit{user} simili e gruppi di \textit{item} simili. Il co-clustering cerca di partizionare simultaneamente le righe (\textit{user}) e le colonne (\textit{item}) della matrice dei \textit{rating} in modo tale che gli \textit{user} all'interno dello stesso co-cluster abbiano comportamenti di valutazione simili e \textit{item} all'interno dello stesso co-cluster siano valutati in modo simile dagli \textit{user} del co-cluster. Il numero di cluster è da definire a priori sia per gli \textit{user} che per gli \textit{item}. Il processo di co-clustering è simile al clustering tradizionale, ma mentre nel clustering classico si raggruppa solo per righe o colonne, nel co-clustering si raggruppano contemporaneamente. 

L'algoritmo, nella fase di inizializzazione, assegna casualmente i co-cluster a \textit{user} e \textit{item}. Durante l'esecuzione i co-cluster vengono aggiornati iterativamente, per cercare di migliorare la qualità del raggruppamento, alternando tra il raggruppamento degli \textit{user} e degli \textit{item} fino a convergenza. Una volta che i co-cluster sono definiti si può calcolare la media dei \textit{rating} all'interno di ciascun co-cluster. $ \overline{C_{ui}} $ rappresenta la media dei \textit{rating} all'interno del co-cluster che contiene lo \textit{user} $ u $ e l'\textit{item} $ i $. In altre parole, è il \textit{rating} medio tra gli \textit{user} e gli \textit{item} che appartengono allo stesso co-cluster.

Si può quindi definire la predizione come

\[
\hat{r}_{ui} = \overline{C_{ui}} + (\mu_u - \overline{C_u}) + (\mu_i - \overline{C_i})
\]

dove $\overline{C_u}$ è la media dei \textit{rating} del cluster di $u$ e $\overline{C_i}$ è la media dei \textit{rating} del cluster di $i$. $ \mu_u - \overline{C_u} $ e $ \mu_i - \overline{C_i} $ vengono definiti \textit{bias}. Se: 
\begin{itemize}
  \item \textit{user} mancante: la previsione è $ \overline{C_i} $
  \item \textit{item} mancante: la previsione è $ \overline{C_u} $
  \item se sia \textit{user} che \textit{item} sono mancanti: la previsione è $ \mu $, la media generale dei rating.
\end{itemize}

I punti di forza dell'algoritmo sono:

\begin{itemize}
  \item Scalabilità: l'algoritmo di co-clustering può essere facilmente parallelizzato, il che lo rende adatto per sistemi di \textit{recommendation} con grandi quantità di dati.
  \item Gestione dei comportamenti complessi: la previsione tiene conto sia del co-cluster, che considera simultaneamente i raggruppamenti di \textit{user} e \textit{item}, sia dei cluster singoli per \textit{user} e \textit{item}. Questo permette di considerare diversità e similarità sia individualmente che insieme, migliorando le previsioni.
  \item Gestione semplice degli aggiornamenti: quando nuovi dati sono aggiunti al sistema, è possibile aggiornare solo i co-cluster rilevanti senza dover ricalcolare tutto da zero. Questo è particolarmente utile per scenari dinamici e sistemi in tempo reale.
  \item Gestione della sparsità: poiché l'algoritmo raggruppa \textit{user} e \textit{item} simili, riduce l'effetto della sparsità permettendo di migliorare la qualità delle previsioni anche quando i dati disponibili sono pochi o incompleti.
\end{itemize}

L'algoritmo soffre anche di diverse problematiche:

\begin{itemize}
  \item Sensibilità ai co-cluster: la qualità delle previsioni dipende molto dal numero di co-cluster scelto e dalla loro qualità. Se il numero di co-cluster è troppo basso, il modello potrebbe non riuscire a catturare le complessità dei dati e fare previsioni imprecise. Se il numero è troppo alto, il modello potrebbe overfittare i dati di addestramento, riducendo la sua generalizzazione. Inoltre, se i co-cluster non sono ben definiti, il modello potrebbe produrre previsioni inaccurate. La scelta iniziale dei co-cluster è fondamentale. Una soluzione è quella di utilizzare l'algoritmo \textit{K-means} per definire la posizione iniziale dei cluster.
  \item Bias del co-cluster: $ \mu_u - \overline{C_u} $ e $ \mu_i - \overline{C_i} $ potrebbero non essere sempre utili in tutte le situazioni. Alcuni \textit{user} o \textit{item} potrebbero avere comportamenti che non sono ben rappresentati dai co-cluster e il modello potrebbe non adattarsi bene a queste situazioni. Ad esempio, \textit{user} che tendono a esprimere \textit{rating} bassi potrebbero non essere gestiti correttamente.
  \item Alto costo computazionale iniziale: l'algoritmo ha un alto costo computazionale durante la fase di addestramento, soprattutto con set di dati molto grandi. Anche se è scalabile, l'ottimizzazione del processo di clustering e la ricerca del numero ottimale di co-cluster richiedono parecchia computazione.
  \item Difficoltà di interpretazione: il co-clustering fornisce gruppi di \textit{user} e \textit{item} che potrebbero non essere sempre facili da interpretare o da analizzare in modo intuitivo.
\end{itemize}

Il costo computazionale per l'addestramento è $ O(W^{\text{glob}} + mkl + nkl) $ dove $ W^{\text{glob}} $ corrisponde al numero di valori diversi da 0 nella matrice in input , $l$ corrisponde al numero di cluster per gli \textit{user} e $k$ corrisponde al numero di cluster per gli \textit{item}.

Per il calcolo della predizione, il costo è $O(1)$, in quanto si tratta di operazioni media e il calcolo dei bias.

Per l'aggiornamento quando un nuovo \textit{rating} viene aggiunto o un nuovo \textit{user}/\textit{item} entra nel sistema, l'algoritmo non ricalcola tutto da zero. Invece, utilizza un aggiornamento incrementale parziale, che si basa sull'aggiornamento delle medie delle matrici:

\begin{itemize}
  \item Se il nuovo \textit{rating} riguarda un \textit{user} e un \textit{item} esistenti si aggiornano direttamente le medie.
  \item Se l'\textit{user} o l'\textit{item} è nuovo, viene assegnato temporaneamente a un co-cluster globale di transizione. Le medie vengono aggiornate e, durante la successiva esecuzione dell'algoritmo, il nuovo \textit{user}/\textit{item} viene riassegnato ai co-cluster regolari.
\end{itemize}

L'aggiornamento ha quindi costo pari a $O(1)$.

\subsection{Slope One}\label{slopeone}

L'algoritmo Slope One, introdotto da Daniel Lemire e Anna Maclachlan~\cite{SlopeOne}, è una delle soluzioni più semplici ed efficienti di collaborative filtering.
%
Le caratteristiche che lo rendono un ottimo algoritmo per la \textit{recommendation} sono:
\begin{itemize}
    \item la semplicità e facilità di implementazione
    \item velocità di calcolo: come verrà presentato più avanti alcuni valori calcolati possono essere salvati e aggiornati all'occorrenza rendendo il calcolo molto più veloce
    \item scalabilità: l'algoritmo può essere abbastanza efficace su dataset di dimensioni moderate, soprattutto se si utilizzano tecniche di compressione dei dati
    \item facilità di interpretazione
\end{itemize}
%
Viene proposto un predittore basato su differenze di \textit{rating} lineari che ha un'efficienza $O(nm)$ per predizione e $O(mn^2)$ per addestramento.

L'algoritmo si basa sulla differenza media tra le valutazioni di due \textit{item} per predire il \textit{rating} mancante. La differenza media dei \textit{rating} di due \textit{item} $i$ e $j$ viene calcolata come:

\[
    \text{dev}(i, j) = \frac{1}{|U_{i,j}|} \sum\limits_{u \in U_{i,j}} (r_{u,i} - r_{u,j})
\]

La matrice simmetrica definita da $\text{dev}(i, j)$ può essere computata una volta e aggiornata velocemente quando vengono aggiunti nuovi dati.

La predizione viene dunque calcolata come:

\[
    \hat{r}_{ui} = \mu_u + \frac{1}{|R_i(u)|} \sum\limits_{j \in R_i(u)} \text{dev}(i, j)
\]

dove:

\begin{itemize}
    \item $R_j = \{ i \mid i \in S(u), i \neq j, |S_{j,i}(\chi)| > 0 \}$ è l'insieme degli \textit{item} rilevanti
    \item $S(u)$ è il sottoinsieme degli item valutati dallo \textit{user} $u$
    \item $S_{j,i}(\chi)$ è l'insieme di tutte le valutazioni $u$ nel dataset $\chi$ che contengono gli \textit{item} $i$ e $j$
\end{itemize}

\begin{figure}[H]
    \centering
    \includegraphics[keepaspectratio]{figures/algorithms/slope_one.PNG}
    \caption{Base dello schema Slope One: le valutazioni dello \textit{user} A di due \textit{item} e la valutazione  dello \textit{user} B di un \textit{item} comune vengono utilizzate per prevedere la valutazione sconosciuta dello \textit{user}.}
    \label{fig:slopeone}
\end{figure}

L'algoritmo soffre anche di diverse problematiche:
\begin{itemize}
    \item sparsità dei dati: le formule presentate prima sono approssimate considerando un dataset non sparso. Nel caso di matrici molto sparse l'algoritmo non sarà in grado di fare previsioni accurate
    \item scalabilità limitata su dataset molto grandi: la memoria necessaria per memorizzare le differenze medie dei \textit{rating} può aumentare rapidamente
    \item non tiene conto nè di personalizzazioni per \textit{user} 
    \item difficoltà a gestire grandi variazioni nelle valutazioni degli utenti
\end{itemize}

L'approccio può essere esteso a modelli ponderati e versioni più avanzate, come per esempio \textit{Weighted Slope One}, che pesa le differenze di \textit{rating} in base alla frequenza di coppie di \textit{item} valutati, e \textit{Regression-based Slope One}, che introduce funzioni non lineari per migliorare la precisione delle previsioni.

\subsection{Evaluation esplicita}\label{evaluation-esplicita}

\section{Algoritmi per il feedback implicito}\label{algoritmi-per-feedback-implicito}

\subsection{ALS (Alternating Least Squares)}\label{als-alternating-least-squares}

\subsection{BPR (Bayesian Personalized Ranking)}\label{bpr-bayesian-personalized-ranking}

\subsection{LMF (Logistic Matrix Factorization)}\label{lmf-logistic-matrix-factorization}

\subsection{Evaluation implicita}\label{evaluation-implicita}

\section{Algoritmi per la similarità item-item}\label{algoritmi-per-la-similarita-item-item}

\section{Algoritmi ibridi}\label{algoritmi-ibridi}

\subsection{LightFM}\label{lightfm}

\section{Introduzione a modelli di Deep Learning}\label{introduzione-a-modelli-di-deep-learning}

\chapter{Analisi}

\section{Analisi del problema}

L'azienda \textit{Data Reply} ha bisogno di sviluppare una libreria che raccolga diversi algoritmi per la \textit{recommendation} di tipo utente-prodotto. Attualmente, le librerie open-source più conosciute in questo ambito, come \textit{Surprise}, \textit{LightFM} e \textit{Implicit}, presentano delle interfacce abbastanza complesse e non standardizzate, che risultano difficili da utilizzare, specialmente per chi non è completamente familiare con il loro funzionamento interno. Questo può causare rallentamenti nello sviluppo dei modelli e difficoltà nell'integrazione dei vari algoritmi nei flussi di lavoro aziendali. Per risolvere questa situazione, l'azienda ha deciso di creare una libreria personalizzata, che permetta di sviluppare modelli di \textit{recommendation} con dimensioni più ridotte rispetto alle soluzioni più moderne e complesse. La motivazione alla base di questa scelta è quella di poter gestire meglio i costi operativi, garantendo comunque buone prestazioni. Inoltre, questa libreria sarà progettata per essere facilmente estendibile, permettendo ai dipendenti di adattarla a esigenze future, come l'integrazione di nuovi algoritmi o il trattamento di diversi tipi di dati. Un ulteriore vantaggio nello sviluppare una libreria interna riguarda il controllo che l'azienda può mantenere sui dati e sulla gestione della privacy. A differenza delle piattaforme esterne, che potrebbero avere politiche di gestione dei dati non completamente trasparenti, lo sviluppo di modelli personalizzati consente all'azienda di avere un maggiore controllo sulla protezione dei dati sensibili e di garantire che vengano rispettate le normative vigenti, come il GDPR. Questo approccio consente anche una maggiore flessibilità nei sistemi e nelle soluzioni che possono essere adottate.

\section{Obiettivi e requisiti del progetto}

\section{Requisiti funzionali}

Di seguito sono elencati i principali requisiti funzionali che la libreria deve soddisfare:

\begin{itemize}
    \item deve seguire lo stile e la struttura di \textit{Scikit-learn}, in modo da garantire un utilizzo intuitivo per gli utenti che già conoscono questa piattaforma
    \item deve includere, come base, algoritmi provenienti da librerie esistenti come \textit{Surprise} e \textit{LightFM}, ma con interfacce standardizzate per semplificare l'uso e renderle più facili da integrare
    \item deve supportare algoritmi per gestire diversi tipi di dati, come feedback espliciti (e.g. valutazioni dirette) e impliciti (e.g. acquisti, visualizzazioni, click)
    \item gli utenti devono poter creare e integrare facilmente modelli personalizzati che siano compatibili con l'interfaccia della libreria e possano sfruttare tutte le funzionalità di \textit{preprocessing} e valutazione già standardizzate
    \item la \textit{coverage} degli \textit{unit test} deve essere almeno del 90\%, per garantire che le funzionalità principali siano ben coperte e funzionanti
\end{itemize}

\section{Requisiti non funzionali}

Per quanto riguarda i requisiti non funzionali, la libreria dovrà:

\begin{itemize}
    \item garantire un'interfaccia semplice ed intuitiva, in modo che gli utenti possano adottarla senza dover conoscere in profondità la parte implementativa del codice
    \item essere modulare, per permettere estensioni e adattamenti a diverse esigenze o casi d'uso
    \item essere ben documentata, con una guida all'uso e esempi pratici che facilitino l'apprendimento e l'integrazione della libreria nei progetti degli utenti
    \item essere compatibile con le versioni più recenti di Python e delle librerie principali come \textit{Pandas}, \textit{NumPy} e \textit{Scikit-learn}
\end{itemize}

In generale, l'obiettivo finale è quello di creare una libreria modulare, estendibile e facile da usare, che rispetti i vincoli aziendali e che, al contempo, risponda alle esigenze specifiche di modelli di \textit{recommendation} meno complessi ma altamente efficaci.

\section{Estensione dei requisiti iniziali}

Partendo dai requisiti iniziali sono state poi proposte soluzioni nuove migliorative:

\begin{itemize}
    \item deve essere incluso un modulo che contiene classi per la manipolazione dei \textit{DataFrame} che permetta operazioni generiche (e.g. rimozione di colonne, normalizzazione e mapping)
    \item viene aggiunto l'utilizzo della libreria \textit{Implicit} per poter coprire una varietà più ampia di casi che utilizzano dataset di feedback impliciti
    \item la libreria deve fornire anche funzioni per analizzare e visualizzare facilmente i dataset, offrendo grafici e statistiche che aiutino a comprendere meglio i dati 
    \item funzionalità extra come la similarità \textit{user-user} o \textit{item-item} permettendo \textit{up-selling}, calcolo del \textit{ranking} per \textit{item} permettendo \textit{item-to-user recommendation}
    \item la possibilità di poter creare versioni custom delle metriche di \textit{evaluation}
\end{itemize}


\chapter{Progettazione}

\section{Architettura}

\section{Progettazione modulo preprocessing}

\section{Progettazione modulo models}

\section{Progettazione modulo evaluation}

\section{Implementazione}\label{implementazione}

\subsection{Implementazione elementi di
preprocessing}\label{implementazione-elementi-di-preprocessing}


\subsection{Interfaccia
RecommendationModel}\label{interfaccia-recommendationmodel}


\subsection{Wrapping dei modelli}\label{wrapping-dei-modelli}


\subsection{Implementazione metriche}\label{implementazione-metriche}


\chapter{Valutazione}

Tutto il codice della libreria è stato soggetto a \textit{unit test}, che sono stati eseguiti con il framework \textit{Pytest}. Non sono stati implementati \textit{integration test}, poiché la libreria è progettata per essere utilizzata in modo modulare e le interazioni tra i componenti sono state verificate singolarmente. In aggiunta ai test è stata calcolata la percentuale di \textit{coverage} raggiunta utilizzando il tool \textit{coverage}. In totale la \textit{coverage} raggiunta è del 100\%. Il modulo di \texttt{visualization} è stato testato manualmente, poiché non è possibile automatizzare completamente la verifica dei grafici generati. Tuttavia, sono stati effettuati controlli visivi per garantire che i grafici fossero corretti e rappresentativi dei dati.

\textit{Dataset utilizzati per i test}:

\begin{itemize}
    \item \textit{LovieLens-1M}: esempio classico di dataset per sistemi di raccomandazione, con 1 milione di valutazioni di film da parte degli utenti. Contiene anche features sui film e sugli utenti, come generi, età e sesso
    \item \textit{Retailrocket Dataset}: presente su \textit{Kaggle}, contiene dati di interazioni tra utenti e \textit{item} in un negozio online, con informazioni su visualizzazioni, aggiunte al carrello e acquisti. È utile per testare modelli di raccomandazione basati su interazioni implicite
    \item \textit{dataset interno}: contiene dati di interazioni tra utenti e polizze assicurative. Le features degli utenti includono età, sesso e nazionalità, mentre le polizze hanno informazioni su tipo, durata e costo. È utile per testare modelli di raccomandazione in ambito assicurativo
\end{itemize}

\section{Test preprocessing}

Per il testing dei \textit{transformers} sono stati utilizzati un misto di dataset reali e sintetici. I dataset sintetici sono stati creati per coprire casi d'uso specifici, come la presenza di valori mancanti o la necessità di utilizzare le varie tipologie di \textit{binning}. I dataset reali sono stati utilizzati per verificare che i \textit{transformers} funzionassero correttamente con dati del mondo reale. Ogni \textit{transformer} è stato testato rispetto alle operazioni equivalenti implementate in \textit{Pandas}, \textit{NumPy} e \textit{Scikit-learn}. I risultati sono stati confrontati per garantire che la libreria producesse gli stessi output. Per i \textit{transformers} che utilizzano possibili modalità come parametri di input (e.g. \texttt{FillNa} con il campo \texttt{method}) sono state usate singole liste o l'oggetto \texttt{ParameterGrid} di \textit{Scikit-learn} per generare tutte le combinazioni possibili di parametri.

\section{Test modelli}

Per i modelli l'obbiettivo dei test è stato in primis di verificare che i modelli fossero compatibili con la libreria \textit{Scikit-learn}. Questa verifica è stata possibile grazie alla funzione \texttt{check\_estimator} di \textit{Scikit-learn}, che ha permesso di testare i modelli rispetto a un insieme di casi d'uso standardizzati. Successivamente, si è proceduto a testare i modelli rispetto al formato dei dati in ingresso e in uscita, per garantire che i modelli potessero essere utilizzati con il formato standard. Infine si è verificato che ciascun modello riproducesse le stesse predizioni dei modelli delle librerie originali a parità di configurazione. Tutti i test sono stati eseguiti utilizzando i dataset di test descritti in precedenza. Per il modello di \textit{LightFM} sono state testate tutte le combinazioni che in includevano o escludevano le features degli utenti e dei \textit{item}.

Per i modelli che implementano \texttt{similar\_items} e \texttt{similar\_users}:

\begin{itemize}
    \item per i modelli di \textit{Implicit} si è fatto un confronto con i metodi della libreria originale, verificando che i risultati fossero equivalenti
    \item per il modello \textit{LightFM} e i modelli di \textit{Surprise} si è verificato che i risultati fossero equivalenti a quelli ottenuti eseguendo il calcolo della similarità \texttt{cosine\_similarity} di \textit{Scikit-learn} o, nel caso specifico del modello \textit{KNNBaseline}, ai risultati ottenuti dalle funzioni di similarità \\ \texttt{surprise.similarities.pearson} \\ o  \texttt{surprise.similarities.pearson\_baseline} e \\ \texttt{surprise.similarities.msd} di \textit{Surprise}
\end{itemize}

Per testare le varie combinazioni di parametri si è utilizzato l'oggetto \\ \texttt{ParameterGrid} di \textit{Scikit-learn}, che permette di generare tutte le combinazioni possibili di parametri per i modelli. Importante sottolineare che alcuni modelli, soprattutto quelli più complessi come \texttt{SVD++}, richiedono un tempo di addestramento significativo, che può variare da pochi secondi a diversi minuti a seconda della configurazione e del dataset utilizzato. Inoltre bisogna considerare che i modelli delle librerie non godono di ottimizzazioni su \textit{GPU} e la libreria \textit{Surprise} nemmeno di ottimizzazioni \textit{Cython} il che la rende lenta nelle fasi di addestramento e predizione. 

\section{Test evaluation}

Per il modulo di \textit{evaluation} sono stati implementati confrontando il valore della metrica rispetto al valore calcolato in modo grezzo, senza l'utilizzo della libreria. In questo modo si è verificato che le metriche calcolate dalla libreria fossero corrette e coerenti con i risultati attesi. Sono state testate anche le metriche per feedback implicito nella loro versione \textit{Factory} con il modello \textit{LightFM}. Le metriche per feedback esplicito in versione \textit{Factory} sono state testate utilizzando come parametro \texttt{predict\_fn} quella base dei modelli di \textit{Surprise} perché non esiste un modello per dati espliciti che potrebbe necessitare di modifiche custom alla metrica. Importante sottolineare che il tempo di calcolo delle metriche dipende sia dalla velocità di predizione del modello sia dalla dimensione del dataset di test. Inoltre, per le metriche implicite la predizione per ciascun utente viene calcolata su tutto l'insieme di \textit{item}, il che può richiedere un tempo significativo, soprattutto per modelli complessi e dataset di grandi dimensioni.

\section{Test altri moduli}
Per tutte le funzioni presenti nei moduli di \texttt{utils} e \texttt{model\_selection} sono state implementate dalle due alle tre funzioni di test sufficienti per coprire i vari casi d'uso.

\section{Usabilità}
La libreria, un volta completata, è stata testata da un gruppo di utenti interni all'azienda \textit{Data Reply} per valutarne l'usabilità e l'efficacia, soprattutto rispetto alle librerie già esistenti ed utilizzate. I feedback raccolti sono stati utilizzati per apportare miglioramenti e ottimizzazioni alla libreria. Si consideri la differenza di linee di codice tra un utilizzo completo di \textit{LightFM} e l'utilizzo della libreria non includendo il preprocessing dei dati:

\begin{lstlisting}[caption=esempio di utilizzo completo della libreria \textit{LightFM}]
RANDOM_STATE = 1234
K = 10
EPOCHS = 100

# train and predict recommendations using original LightFM model
lightfm_model = lightfm.LightFM(random_state=RANDOM_STATE)
# fit Dataset based on features and data
lightfm_dataset = Dataset()
item_features_data = unique(list([(row['item_id'], f"{col}:{value[col]}") 
                                  for _, value in policies_item.iterrows() 
                                  for col in ['duration', 'cost']]))
user_features_data = unique(list([(row['user_id'], f"{col}:{row[col]}")
                                  for _, row in policies_users.iterrows() 
                                  for col in ['age', 'sex', 'nationality']]))

lightfm_dataset.fit(users=unique(policies_users['user_id']),
                    items=unique(policies_users['item_id']),
                    item_features=item_features_data,
                    user_features=user_features_data)

# create interactions matrix and weights matrix instance for data
interactions, weights = lightfm_dataset.build_interactions(trainset.values)

# train the model with features
lightfm_model.fit(interactions,
                  sample_weight=weights,
                  item_features=item_features,
                  user_features=user_features,
                  epochs=EPOCHS)

# get mapping
user_id_mapping, _, item_id_mapping, _ = lightfm_dataset.mapping()

# creates all ids for prediction
predictions_user_ids = np.array([user_id_mapping[user] 
                                 for user in testset['user_id']])
unchanged_user_ids = [user for user, _ in id_pairs]
predictions_item_ids = np.array([item_id_mapping[item] 
                                 for item in testset['item_id']])
unchanged_item_ids = [item for _, item in id_pairs]

# prediction using LightFM model
lightfm_scores = lightfm_model.predict(predictions_user_ids,
                                       predictions_item_ids,
                                       item_features=item_features,
                                       user_features=user_features)
# zip together unchanged ids and predictions scores
total_prediction = [{'user_id': user_id, 'item_id': item_id, 'interaction_weight': score}
                    for user_id, item_id, score in zip(unchanged_user_ids, unchanged_item_ids,lightfm_scores)]
sort_function = lambda x: (x['interaction_weight'], x['item_id'])
map_function =


def select_predictions(grouped_predictions):
    # sort predictions by ranking
    sorted_predictions = sorted(grouped_predictions, 
                                key=lambda x: (x['interaction_weight'], x['item_id']), 
                                reverse=True)
    # select first k
    k_predictions = sorted_predictions[:K]
    # extract only ids
    mapped_predictions = map(lambda x: x['item_id'], k_predictions)
    return list(mapped_predictions)

# compute groups
groups = {key_id: group for key_id, group in 
          groupby(predictions, lambda x: x['user_id']).items()}

# group by key, and extract for each id top k predictions
rankings = {key_id: select_predictions(grouped_predictions)
            for key_id, grouped_predictions in groups.items()}
\end{lstlisting}

\begin{lstlisting}[caption=esempio di utilizzo del modello \texttt{LightFM} libreria]
RANDOM_STATE = 1234
K = 10
EPOCHS = 100

# create model
new_lightfm_model = LightFM(random_state=RANDOM_STATE)
# train model with
new_lightfm_model.fit(trainset, 
                      item_features=item_features,
                      user_features=user_features, 
                      epochs=EPOCHS)
# create predictions
new_total_prediction = new_lightfm_model.predict(testset, 
                                                 user_features=user_features,
                                                 item_features=item_features)
# create rankings from predictions
new_rankings = create_rankings_from_implicit(dataset=testset,
                                             scores=new_total_prediction, 
                                             k=K, 
                                             is_user_ranking=True)
\end{lstlisting}

Come si può notare, l'utilizzo della libreria consente di ridurre significativamente il numero di linee di codice necessarie per addestrare e utilizzare un modello di \textit{LightFM}, rendendo il codice più leggibile e mantenibile.

\section{Documentazione}
Per garantire la facilità d'uso della libreria, è stata creata una documentazione completa utilizzando \textit{Sphinx} utilizzando il tema \texttt{sphinx\_rtd\_theme}. La documentazione include:

\begin{itemize}
    \item una guida all'installazione e all'utilizzo della libreria
    \item una descrizione dettagliata di ogni modulo e delle sue funzioni
    \item esempi di utilizzo per ciascun modulo
\end{itemize}

\section{CI/CD}

Per garantire la qualità del codice e l'affidabilità della libreria, è stato implementato un sistema di \textit{Continuous Integration/Continuous Deployment} (\textit{CI/CD}) utilizzando la \textit{GitHub Actions} \texttt{astral-sh/setup-uv@v5} per la gestione del tool \textit{uv}. Questo sistema esegue automaticamente i test ogni volta che viene effettuata una modifica con versione del codice, notificando con una mail in caso di fallimento. Si è deciso di utilizzare l'esecuzione dei test solamente per modifiche significative in quanto il tempo di esecuzione dei test è elevato. Inoltre, ha il compito di generare la documentazione della libreria e di pubblicarla su \textit{GitHub pages} e su \textit{Confluence} utilizzando un'integrazione ad hoc.

\section{Limiti della soluzione proposta}

La libreria è stata progettata per essere modulare e flessibile, ma presenta alcuni limiti:

\begin{itemize}
    \item non supporta modelli basati su \textit{deep learning} come \textit{Neural Collaborative Filtering}
    \item tutto il codice è scritto in \textit{Python} nativo o in \textit{Cython}, quindi non godendo di ottimizzazioni su \textit{GPU} può limitare le prestazioni su dataset di grandi dimensioni. Questo potrebbe rendere l'addestramento, la predizione e i test molto lenti, soprattutto per modelli complessi
    \item gli algoritmi proposti, soprattutto quelli di \textit{Surprise}, hanno difficoltà a gestire dataset di dimensioni grandi. Inoltre, utilizzano solamente le informazioni di interazione tra utenti e \textit{item}, senza considerare altre informazioni sull'interazione o le features degli utenti e degli \textit{item} (tranne \textit{LightFM}) che potrebbero migliorare le prestazioni del modello
    \item non offre nessuna funzionalità per calcolare la similarità tra \textit{item} tramite immagini o testi, che sono spesso utilizzate nei sistemi di raccomandazione moderni    
\end{itemize}




% \chapter{Conclusioni e Lavori Futuri}
Questo lavoro di tesi si è concentrato sul tema dei sistemi di \textit{recommendation}, approfondendone sia gli aspetti teorici sia quelli pratici. L'obiettivo principale è emerso da un'esigenza concreta riscontrata durante un tirocinio svolto presso \textit{Data Reply}: semplificare e uniformare l'utilizzo delle librerie open-source più diffuse (come \textit{Surprise}, \textit{LightFM} e \textit{Implicit}), spesso caratterizzate da interfacce disomogenee e poco standardizzate. Per rispondere a questa esigenza, è stata sviluppata una libreria in \textit{Python} finalizzata a semplificare la creazione, il test e l'utilizzo di modelli di \textit{recommendation} in contesti produttivi.

La prima parte della tesi presenta una panoramica generale sui sistemi di raccomandazione, partendo dagli algoritmi più tradizionali fino ad arrivare a tecniche più recenti basate sul \textit{Deep Learning}. Particolare attenzione è stata dedicata alla distinzione tra modelli basati su feedback \textit{espliciti} e \textit{impliciti}, evidenziandone vantaggi, limiti e differenze principali. Tra gli algoritmi trattati, sono stati approfonditi in particolare i modelli basati su \textit{K-Nearest Neighbors (KNN)} e le diverse tecniche di \textit{Matrix Factorization}. Per facilitare il confronto tra modelli, è stata introdotta una notazione condivisa e una tassonomia coerente.

La seconda parte descrive lo sviluppo della libreria realizzata durante il tirocinio. L'obiettivo era creare uno strumento compatibile con \textit{Scikit-learn}, come richiesto dall'azienda. La libreria è stata progettata in modo modulare, con componenti dedicate a \texttt{preprocessing}, \texttt{models}, \texttt{model\_selection}, \texttt{evaluation}, \texttt{utils} e \texttt{visualization}, risultando compatibile sia con \textit{NumPy array} che con \textit{Pandas DataFrame}. Per facilitare l'integrazione di librerie esterne, sono stati sviluppati \textit{wrapper} uniformi per \textit{Surprise}, \textit{Implicit} e \textit{LightFM}. La gestione delle dipendenze è stata semplificata mediante l'utilizzo di \texttt{uv}, mentre i test automatici sono stati implementati con \textit{Pytest}, raggiungendo una copertura del 100\% (ad eccezione del modulo \textit{visualization}, testato manualmente). I test sono stati condotti su diversi dataset: \textit{MovieLens-1M}, \textit{Retailrocket Dataset} e un dataset interno aziendale, dimostrando la flessibilità della libreria in scenari eterogenei.

La libreria è stata inoltre presentata in un workshop interno, ricevendo riscontri positivi. In particolare, i moduli \texttt{visualization} e \texttt{preprocessing} sono stati effettivamente riutilizzati in altri progetti aziendali separati, a conferma della loro utilità pratica.

Nonostante i risultati ottenuti, la libreria presenta ancora alcuni limiti. Attualmente non include modelli basati su \textit{Deep Learning} (come il \textit{Deep Factorization Machines} o tramite l'utilizzo di più moderne architetture a \textit{Tranformers} come \textit{Transformers4Rec}) e non è ottimizzata per l'uso della \textit{GPU}, aspetto rilevante in presenza di dataset di grandi dimensioni. Inoltre, alcune librerie integrate (come \textit{Surprise}) non supportano l'utilizzo di \textit{feature} aggiuntive e gestiscono in modo inefficiente dataset di grandi dimensioni, fatta eccezione per \textit{LightFM} che offre un supporto parziale. La libreria non prevede infine funzionalità per il calcolo della similarità tra \textit{item} basata su contenuti testuali o visivi, elementi sempre più centrali nei moderni sistemi di raccomandazione.

Tra gli sviluppi futuri possibili si segnalano l'estensione del supporto a modelli basati su \textit{Deep Learning} presentati, l'abilitazione all'utilizzo della \textit{GPU}, e l'integrazione di \textit{feature} contestuali relative a utenti e \textit{item}. Ulteriori miglioramenti includono l'aggiunta di moduli per la similarità basata su testi e immagini, nonché il supporto a tecniche di \textit{association rule mining} (come \textit{FP-Growth} o \textit{ECLAT}), con l'obiettivo di rendere la libreria più completa e adatta a scenari moderni e complessi.



%----------------------------------------------------------------------------------------
% BIBLIOGRAPHY
%----------------------------------------------------------------------------------------

\backmatter

\bibliographystyle{alpha}
\bibliography{bibliography}

\begin{acknowledgements}
Optional. Max 1 page.
\end{acknowledgements}
\end{document}
