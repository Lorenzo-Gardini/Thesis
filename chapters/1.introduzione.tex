\chapter{Introduzione}

I sistemi di \textit{recommendation} sono elementi fondamentali in molti settori legati alla vita quotidiana, come l'e-commerce, i social network e i servizi di streaming, dove la personalizzazione dei contenuti è volta al miglioramento dell'esperienza dell'utente. L'efficacia di questi sistemi è strettamente legata alla disponibilità e al volume crescente dei dati che, se opportunamente sfruttati, possono migliorare la qualità delle raccomandazioni.

Nella prima parte di questa tesi si fornirà una panoramica generale su questi sistemi, esplorando i principali approcci e le tecniche impiegate per realizzare modelli di raccomandazione efficaci. Si tratteranno i modelli basati su feedback espliciti e impliciti, esaminando le caratteristiche e le differenze tra i due tipi di dati. Successivamente, si approfondiranno alcuni algoritmi specifici, esplorando i loro vantaggi, svantaggi e le loro applicazioni in vari contesti.

La seconda parte della tesi viene descritta la parte inerente al lavoro svolto durante il mio tirocinio svolto presso l'azienda \textit{Data Reply}, in cui mi sono concentrato sulla creazione di una libreria in Python che facilitasse la creazione, il testing e l'utilizzo modelli di raccomandazione. Nello specifico, ho lavorato sul wrapping delle interfacce dei modelli di librerie popolari come \textit{Surprise}, \textit{LightFM} e \textit{Implicit}, con l'obiettivo di rispettare i vincoli specifici imposti dalla mia azienda. Verranno discusse le sfide affrontate, le soluzioni adottate e i risultati ottenuti.

Il contributo di questa tesi si inserisce nel contesto della crescente necessità di rendere i sistemi di raccomandazione sempre più accessibili e facilmente integrabili in ambienti di lavoro aziendali, dove personalizzazione ed efficienza sono essenziali.
