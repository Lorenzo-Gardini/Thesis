\chapter{Conclusioni e Lavori Futuri}
Questo lavoro di tesi si è concentrato sul tema dei sistemi di \textit{recommendation}, approfondendone sia gli aspetti teorici sia quelli pratici. L'obiettivo principale è emerso da un'esigenza concreta riscontrata durante un tirocinio svolto presso \textit{Data Reply}: semplificare e uniformare l'utilizzo delle librerie open-source più diffuse (come \textit{Surprise}, \textit{LightFM} e \textit{Implicit}), spesso caratterizzate da interfacce disomogenee e poco standardizzate. Per rispondere a questa esigenza, è stata sviluppata una libreria in \textit{Python} finalizzata a semplificare la creazione, il test e l'utilizzo di modelli di \textit{recommendation} in contesti produttivi.

La prima parte della tesi presenta una panoramica generale sui sistemi di raccomandazione, partendo dagli algoritmi più tradizionali fino ad arrivare a tecniche più recenti basate sul \textit{Deep Learning}. Particolare attenzione è stata dedicata alla distinzione tra modelli basati su feedback \textit{espliciti} e \textit{impliciti}, evidenziandone vantaggi, limiti e differenze principali. Tra gli algoritmi trattati, sono stati approfonditi in particolare i modelli basati su \textit{K-Nearest Neighbors (KNN)} e le diverse tecniche di \textit{Matrix Factorization}. Per facilitare il confronto tra modelli, è stata introdotta una notazione condivisa e una tassonomia coerente.

La seconda parte descrive lo sviluppo della libreria realizzata durante il tirocinio. L'obiettivo era creare uno strumento compatibile con \textit{Scikit-learn}, come richiesto dall'azienda. La libreria è stata progettata in modo modulare, con componenti dedicate a \texttt{preprocessing}, \texttt{models}, \texttt{model\_selection}, \texttt{evaluation}, \texttt{utils} e \texttt{visualization}, risultando compatibile sia con \textit{NumPy array} che con \textit{Pandas DataFrame}. Per facilitare l'integrazione di librerie esterne, sono stati sviluppati \textit{wrapper} uniformi per \textit{Surprise}, \textit{Implicit} e \textit{LightFM}. La gestione delle dipendenze è stata semplificata mediante l'utilizzo di \texttt{uv}, mentre i test automatici sono stati implementati con \textit{Pytest}, raggiungendo una copertura del 100\% (ad eccezione del modulo \textit{visualization}, testato manualmente). I test sono stati condotti su diversi dataset: \textit{MovieLens-1M}, \textit{Retailrocket Dataset} e un dataset interno aziendale, dimostrando la flessibilità della libreria in scenari eterogenei.

La libreria è stata inoltre presentata in un workshop interno, ricevendo riscontri positivi. In particolare, i moduli \texttt{visualization} e \texttt{preprocessing} sono stati effettivamente riutilizzati in altri progetti aziendali separati, a conferma della loro utilità pratica.

Nonostante i risultati ottenuti, la libreria presenta ancora alcuni limiti. Attualmente non include modelli basati su \textit{Deep Learning} (come il \textit{Deep Factorization Machines} o tramite l'utilizzo di più moderne architetture a \textit{Tranformers} come \textit{Transformers4Rec}) e non è ottimizzata per l'uso della \textit{GPU}, aspetto rilevante in presenza di dataset di grandi dimensioni. Inoltre, alcune librerie integrate (come \textit{Surprise}) non supportano l'utilizzo di \textit{feature} aggiuntive e gestiscono in modo inefficiente dataset di grandi dimensioni, fatta eccezione per \textit{LightFM} che offre un supporto parziale. La libreria non prevede infine funzionalità per il calcolo della similarità tra \textit{item} basata su contenuti testuali o visivi, elementi sempre più centrali nei moderni sistemi di raccomandazione.

Tra gli sviluppi futuri possibili si segnalano l'estensione del supporto a modelli basati su \textit{Deep Learning} presentati, l'abilitazione all'utilizzo della \textit{GPU}, e l'integrazione di \textit{feature} contestuali relative a utenti e \textit{item}. Ulteriori miglioramenti includono l'aggiunta di moduli per la similarità basata su testi e immagini, nonché il supporto a tecniche di \textit{association rule mining} (come \textit{FP-Growth} o \textit{ECLAT}), con l'obiettivo di rendere la libreria più completa e adatta a scenari moderni e complessi.

